%!TEX root = ../ClassicThesis.tex

\chapter{Background}
\label{ch:bg}

%------------------------------------------------------------------------------
\section{Information Theory, and its applications within Neuroscience}
\label{sec:bgit}

A common experimental methodology used in neuroscience is to record the extracellular activity of a single neuron under different conditions.
From this, we can compare the activity of the neuron under the different conditions to examine whether it is dependent on this set of conditions, and if so investigate the nature of the relationship between the two.

Frequently, the approach used is to take many recordings of the same neuron for the same condition, and then take the average across these repetitions (``trials'') to reduce the effects of neuronal variability, producing a \ac{PSTH}, for instance.
This neuronal variability is often referred to as ``noise'', however it is debatable as to whether differences in the behaviour of individual neurons between trials are due to noise within the system or are in fact representative of changes in other, as-yet unknown, hidden or latent variables within the system (see \S\ref{sec:bg-noisecorr} for further discussion).

Such a simple treatment of the data --- averaging the over response over repetitions --- is fundamentally flawed, since this is not the manner in which brains process stimuli.
At any moment in time, the brain has access to the activity of many neurons simultaneously (not a single neuron in isolation), but only has a single sample of each one (not multiple instantiations of the same neuron).

If we instead use information theory to study the neuronal activity, we can consider how much information there is across a system containing multiple neurons during an isolated period of time, for instance a single trial.
By using an information theoretic technique, we can overcome the limitations of the more simple methods; but no method is perfect and there are other limitations which arise when using information theory instead.
% CITE no free lunch theorem
In this section, I will first outline the analytic procedure through which information theoretic analysis is applied to neuroscientific data, some of the problems which arise, and how to try to overcome them.

In the context of trying to experimentally investigate properties of the sensory cortex of the brain, one typically uses an experimental set-up with a finite collection of discrete experimental stimuli.
These stimuli are then repeatedly presented to the sensory organ in an appropriate fashion, and the responses during each presentation are recorded.

Even when investigating the responses to stimulus features which vary continuously, such as line orientation for the visual stimulation and frequency for auditory stimulation, one has two options.
Firstly, the features can be preserved in a continuous space and the information about the stimulus estimated using Fisher information. The inverse of the Fisher information provides a lower bound on the mean squared error which any decoder can achieve \citep{Quiroga2009}.
Alternatively, the stimulus space can be discretized into a finite subset of stimuli from which we can draw samples with repetition.
Once the samples are discretized and presented repeatedly, we can use Shannon information to compute the amount of information about the stimulus identity contained in the neuronal response.
If the stimulus set is intrinsically discrete, Fisher information can not be used and an information theoretic analysis can only be performed using Shannon information.

For such an experimental set-up, let us assume that on each trial the stimulus $s$ is selected at random with probability $P(s)$ from a set of discrete stimuli $\SET{S}$, containing $S$ unique stimuli.
The neuronal response could be one (or more than one) of several data types, such as a spike train from one or more neurons, the \ac{LFP}, \ac{CSD}, \ac{BOLD} signal, a calcium indicator, \ac{EEG}, or others \citep{Magri2009,Quiroga2009}.
The principles of information theory can be applied whichever neural signal is taken to be the response.
In Chapter~\ref{ch:pl}, we will work with information encoded in \ac{MUA} and spike trains, whilst in Chapters~\ref{ch:lam}, \ref{ch:xlam} and \ref{ch:plam} we will be considering the \ac{MUA}, \ac{LFP} and \ac{CSD}.

With regards to the analysis of sensory recordings (with which this thesis will be concerned), the different conditions used on the trial are typically different stimuli, and the extracellular recordings provide us with the neuron's response to the stimuli.
When applying information theory to neuronal data, we treat the brain as a communication channel, transmitting information about sensory input.
We are hence interested in how much information the response in the brain contains about which stimulus was presented to it.

However, it should be noted that we frame the problem in the context of a \textit{communication channel} simply because this is the framework around which Shannon information is formulated \citep{mackay2003information}.
Within information theory, systems are modelled with information passing between a transmitter and receiver through a communication channel.
The message passing between them in modified as it passes through the channel, and the receiver must attempt to decipher which message was originally sent.

In some ways, the function of the brain is similar to that of a compression algorithm.
The initial encoding of the stimulus as transcribed by the appropriate sensory organ contains a large amount of information about the precise input stimulus --- for example the individual pixel values with an image stimulus --- which has a large amount of redundancy if one is interested only in detecting, classifying, and reacting to stimuli.
A binary image of only $17 \times 17$ pixels can express \num{9.9e86} different states --- around a million times more than there are atoms in the visible universe, thought to be around \num{1e80}.
However the vast majority of these images (for this, or equally true for a larger image with more intensity levels and colours) resemble unstructured random noise.
The set of images which are of interest for interacting with a real world environment is vastly smaller, and with an appropriate high-level statistical model of the appropriate subset of the class of all potential stimuli can be compressed down to a much smaller number of bytes.
For instance, whether a given visual stimulus contains the face of familiar person.

After a stimulus has been processed by the brain, information about the exact intensities of individual pixels is lost, but salient information about the environment is preserved.
We can hence investigate how stimuli are encoded within the brain by computing the amount of information about properties of the stimulus contained within the neural recordings.
Here, we make the following assumption: if the neuronal activity is observed to contain information about the stimulus, we can assume this information is present due to the manner within which information is encoded by the brain, and that this information can be drawn upon to inform decisions taken with regard to the stimulus.
We rationalise this assumption on the basis that we know the brain contains information about stimuli (otherwise it would be functionally blind/deaf), and it would be wasteful expend resources encoding stimuli accurately but in a non-functional manner.

The neural data which can be collected with modern experimental equipment is very dense and rich in content.
For instance, individual spikes can be recorded with the precision of fraction of a millisecond, and broadband \acp{LFP} allow for many frequency components to be analysed from the same recording.
Typically, it is not possible to compute the information about the stimulus contained in the entire data stream all at once.
The reason for this is the relatively small number of trials which can be collected for any given dataset.

The repetitions of presenting the same stimulus and recording the neural response must be performed with the experimental setup the same, and recording taken from the same neurons.
However, there is a maximum duration for any \invivo experiment of a few hours, after which the experimenters and subjects will be fatigued.
This, naturally, leads to a maximum number of repetitions of an experimental protocol which can be completed within this timeframe, and that is in the order of 100.
Should more trials be undertaken in a different experimental session, there is no guarantee that the neurons recorded from will be the same, so the response structure may be different and can't be combined with that of the prior session.

Using information theory, we can investigate the nature of the neural code used by individual neurons and populations of neurons \citep{Optican1987}.
For example, if our dataset consists of recordings of neuronal spiking activity, we can consider the amount of information contained in the spike train coincident with a \SI{40}{\milli\second} stimulus, say.
First, we can consider our response vector to be the total number of spikes over the \SI{40}{\milli\second} window and compute the information contained in these about the identity of the presented stimulus.
Second, we can consider our response vector to be the number of spikes in each quarter of the stimulus presentation period (four \SI{10}{\milli\second} windows).
This step could equally be performed with more windows of finer granularity, so in general we would have a response vector $r = [r_1, \ldots, r_L]$, with $L$ windows each of length $\nicefrac{T}{L}$ and $r_i$ the number of spikes during the $i$-th window\footnote{In our example $T = \SI{40}{\milli\second}$.}.
Since the information contained in single \SI{40}{\milli\second} window approach is, by construction, contained in the vector of responses within the shorter windows, we can investigate amount of information contained within the timing of the spikes.
If there is no significant difference between the amount of information about the stimulus contained in the two vectors, it seems reasonable to conclude that the stimulus, or some attributes which distinguish it, are encoded in the firing rate, whilst the exact timing of the spikes is unimportant.

% Typically \citep{Quiroga2009,Brasselet2012,Panzeri2007,Arabzadeh2006,Strong1998}, a certain post-stimulus time window of duration $t$ in \SIrange{20}{40}{\milli\second} is chosen and a neural code is constructed which forms a discrete, multi-dimensional array $r = \{r_1, \ldots, r_L\}$ of dimension $L$, because a discrete response of this form is appropriate for performing an information theoretic analysis on.
% To look at the information given by the firing rate of a single neuron, we might use a spike-count code where $L=1$ and $r$ is equal to the number of spikes elicited in the window $t$.
% Similarly, to look at the information contained in the firing rate of a many neurons, we would use a spike-count code where $r_i$ is equal to the number of spikes elicited in the window $t$ for spike train of the \nth{i} neuron, and let $L$ equal the number of neurons to be studied.
% In comparison, to look at the information contained in the millisecond level spike timing of a single neurons, we would divide the time window into $L$ bins of length $\Delta t = \nicefrac{t}{L}$ such that $\Delta t$ is the assumed time precision of the code, and set $r_i$ to be the number of spikes elicited in \nth{i} time-bin.

In general, we will choose some framework through which the raw data is reduced to a manageable finite set of possible states, $\SET{R}$.
Having constrained both our encoding of the stimulus and the response to a finite set of states, we can investigate the relationship between them using Shannon information \citep{Shannon1948}.

Within this theory, information is quantified in a manner analogous to how ``surprised'' a receiver would be if they were to reveal the contents of a message sent by the transmitter.
Unless there is only one possible message, there is uncertainty over what will be sent, potentially with some messages more likely than others.
If an \textit{a priori} likely message is received, this confirms the expectations so the receiver, so they are less ``surprised''.
If an unlikely message is received, the receiver is more ``surprised''.
Intuitively, the amount of information gained on receipt of the message is related to how much the uncertainty in the message was reduced upon its arrival.

Rigorously, we define

The relationship between the distribution of responses and stimuli is evaluated by first quantifying the variability of the responses.
This can be done with the entropy \citep{Shannon1948} of the responses, and we define the \textit{response entropy} to be
\begin{equation}
H( \SET{R} )
= - \sum_{r} P(r) \log_2 P(r)
\end{equation}
where $P(r)$ is the probability of observing the response $r$ on any trial regardless of the stimulus.
However, the responses given by neurons are ``noisy'', so they do not give the same response on every trial even if the same stimulus is presented.
Consequently, we must also consider the variability due to noise by computing the \textit{noise entropy}, defined as
\begin{equation}
H( \SET{R} | \SET{S} )
= - \sum_{r,s} P(s) P(r|s) \log_2 P(r|s)
.\end{equation}
The information about the stimulus which is transmitted in the response is then given by the difference of these, and the \textit{mutual information} between stimulus and response is given by
\begin{equation}
I( \SET{S} ; \SET{R} )
= H( \SET{R} ) - H( \SET{R} | \SET{S} )
= \sum_{r,s} P(r,s) \log_2 \frac{P(r|s)}{P(r)}
.\end{equation}
The mutual information can conceptualised how much an independent observer can expect their uncertainty in the stimulus $s$ presented on a single trial to be reduced by if they were to observe the neural response.
When using base two logarithms, the mutual information is measured in bits, where gaining \SI{1}{bit} of information about something means a halving in uncertainty about it.
The mutual information is zero if and only if responses are completely independent of stimuli.
We will frequently abbreviate mutual information to just information.
% CITE MacKay

When working with experimental data, the probabilities $P(s)$, $P(r)$ and $P(r|s)$ must be estimated from the available data.
This presents a major problem, because precise values for the probabilities can only be found exactly from their frequencies in the data if there is an infinite amount number of trials available, and real-world experiments (somewhat inconveniently) contain only a limited number of trials.
The estimated probabilities are subject to statistical error, leading to an associated systematic error (bias) and statistical variance in the estimates of the entropies and mutual information.
The bias in particular is an issue, causing the mutual information to be upwardly biased, which can lead to incorrect conclusions if not corrected.
Conceptually, this is because finite sampling can lead to spurious differences in the response distributions, making them seem more discriminable that they really are.
We refer to the bias uncorrected mutual information as $I_{\text{uncorrected}}(\SET{S};\SET{R})$.

Fortunately, several techniques exist to correct for the bias.
Several of these bias correction methods focus on expanding out the measured information as a power series \citep{Miller1955,Treves1995} in terms of $\nicefrac{1}{N}$, where $N$ is the number of trials in the dataset, though technically the relationship with the power series only holds in the asymptotic sampling regime with a very large number of trials.
The first term in the bias, proportional to $\nicefrac{1}{N}$, has a coefficient which depends only on the number of stimuli, $S$, and possible responses $\overline{R}$ .
However, $\overline{R} \neq R$ because many responses which are theoretically possible by the construction of the response code may be in fact be impossible to generate.
Furthermore, $\overline{R}$ cannot be found simply by looking at the number of unique responses in the dataset, since low probability responses may not have occurred in the finite number of trials sampled, leading to an underestimation of $\overline{R}$.
Consequently, one approach to bias correction, \iac{PT} \citep{Panzeri1996}, uses a Bayesian procedure to estimate the true number of possible responses and then subtracts the leading term of the bias from the mutual information.
A second method of correcting for the bias which makes use of the power series expansion is \iac{QE} method \citep{Strong1998}.
Here, the uncorrected mutual information is assumed to be well approximated by
$$
I_{\text{uncorrected}}(\SET{S};\SET{R}) = I_{\text{true}}(\SET{S};\SET{R}) + \frac{a}{N} + \frac{b}{N^2}
,$$
with the free parameters $a$ and $b$ found by computing the information content with fractions of the full available dataset (\ie{} using $\nicefrac{N}{2}$ and $\nicefrac{N}{2}$ trials).

Other bias correction methods which do not utilise the power series expansion include \iac{NSB}.
This uses a Bayesian inference approach to entropy estimation with a specially chosen prior probability distribution to make the prior expected entropy distribution uniform \citep{Nemenman2004}.

A completely different method of bias correction which works for responses with dimension $L > 1$, is to compute an estimate of the information known as $I_{\text{sh}}$.
This is found \citep{Montemurro2007} by computing two additional terms: $H_\text{ind}( \SET{R} | \SET{S} )$, the noise entropy estimate if the dimensions of $r$ are assumed to be independent from one another; and $H_\text{sh}( \SET{R} | \SET{S} )$, the noise entropy estimate when bins are shuffled along each dimension, $r_i$, of the response to generate pseudo-response arrays.
The advantage of this is $H_\text{ind}( \SET{R} | \SET{S} )$ and $H_\text{sh}( \SET{R} | \SET{S} )$ should be equal, but $H_\text{sh}( \SET{R} | \SET{S} )$ has a bias around the same magnitude as $H( \SET{R} | \SET{S} )$, so we can compute
$$
I_{\text{sh}}(\SET{S};\SET{R}) = H( \SET{R} ) - H_\text{ind}( \SET{R} | \SET{S} ) + H_\text{sh}( \SET{R} | \SET{S} ) - H( \SET{R} | \SET{S} )
,$$
which has a much smaller bias than $I(\SET{S};\SET{R})$.

Both \iac{PT} and \ac{QE} bias correction methods give similar approximations to the true information, whilst \ac{NSB} outperforms them with a less biased estimate \citep{Panzeri2007}.
However, \ac{NSB} is very computationally intensive \citep{Panzeri2007}, so we will not be making use of it in the presented body work.
All these bias correction methods tend to make a trade off between variability and bias, introducing more terms and hence more variability to reduce the size of bias term.

% rule of thumb number of trials per stimulus required for decent estimate in each case


%------------------------------------------------------------------------------
% \section{Applying Information Theory to Neuroscience data}

% Previous research has looked at the information in the onset transient in \ac{V1}.
% There lower variability in the transient response, and this gives the most information about the stimuli.
% Adding activity from later is not useful.
% \citep{Muller2001}


%------------------------------------------------------------------------------

\subsection{Noise Correlations}
\label{sec:bg-noisecorr}

When an individual is repeatedly presented with the same stimulus, a representation of the stimulus is formed within the brain of the individual.
One might expect that, should we eliminate variations in the environment such that an identical audio track is played without any background stimulus or a visual image is presented with the eyes held in place, the activity within the associated sensory cortex would be identical on each repetition of the stimulus presentation.

Whilst the information encoded across the whole population of neurons may indeed be the same, the activity of each individual neuron varies with each presentation of the stimulus. Both the number of spikes evoked (or firing rate) and the timing of these spikes varies for each neuron. 

One can imagine the activity of each neuron fluctuating around a mean (though not with a Gaussian distribution). Each neuron has a certain expected value for its activity in response to any stimulus which could be presented. For classes of stimuli for which the neurons are tuned, such as gratings of different orientations or different contrasts, we can plot a tuning curve mapping the parameter space describing the stimuli to a particular response activity, specific to each neuron.

Noise correlations are observed when the responses to fixed stimuli co-vary for two neurons. That is to say, positive noise correlations are found when two neurons tend to have higher-than-expected activity on the same trials as one another (and lower-than-expected on the same trials too).

Intuitively, one can see that such noise correlations between pairs of neurons can inhibit the accuracy with which the stimulus is encoded in their activities. Suppose that two neurons both respond monotonically more to stimuli of higher contrast. Knowing their tuning curves and their current activity, we can decode the contrast of the current stimulus with some level of accuracy. If the two neurons are independent of one another, knowing the activity of both will give us a more accurate and more precise estimate of the actual contrast of the stimulus. But if the activity the pair of neurons is positively correlated, the information conveyed from the pair of neurons is reduced --- when one gives an overestimate of the contrast from a by-chance elevated activity level, so does the other. In contrast, negative correlations would enhance our decoding accuracy, for an overestimate from one neuron would more frequently be mitigated by an underestimate from the other.

For a long period of time, neuroinformatians had argued that positive noise correlations are always detrimental to the encoding of stimuli. However, more recent theoretical work has found that, when considering a non-homogeneous population of neurons --- a population where the tuning curve is not the same for each neuron --- positive noise correlations can in fact be beneficial to the encoding of information about the stimulus.

This can be envisioned by extending the 1-D mapping between a stimulus parameter and the activity of a single neuron to a 2-D plane of pairwise activity levels which is traversed as the stimulus parameter is adjusted. For orientation tuning (where the angular parametrisation of the stimulus is cyclic), tracing out the expected activity of a pair of neurons in response to the stimulus class will result in a closed loop in the 2-D plane. For other stimuli, which are non-cyclic, the trace is an open curve.

Deviations to the response of the pair of neurons which move in the direction of the curve is detrimental to stimulus decoding, whereas deviations which are tangential to the curve are beneficial (better even than independence). Whether these directions correspond to positive or negative correlations depends on the direction of the line.
