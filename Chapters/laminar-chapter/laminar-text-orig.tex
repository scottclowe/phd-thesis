% This file was converted to LaTeX by Writer2LaTeX ver. 1.4
% see http://writer2latex.sourceforge.net for more info
\documentclass{article}
\usepackage[ascii]{inputenc}
\usepackage[LGR,T1]{fontenc}
\usepackage[greek,english]{babel}
\usepackage{amsmath}
\usepackage{amssymb,amsfonts,textcomp}
\usepackage{array}
\usepackage{supertabular}
\usepackage{hhline}
\makeatletter
\newcommand\arraybslash{\let\\\@arraycr}
\makeatother
\setlength\tabcolsep{1mm}
\renewcommand\arraystretch{1.3}
\title{}
\begin{document}
\title{Lamina and Frequency Specific Distribution of Information in Primary Visual Cortex}
\maketitle

Scott C. Lowe,1,*Daniel Zaldivar,2 Yusuke Murayama,2 Mark C. W. van Rossum,1 Nikos K. Logothetis,2, 3 and Stefano Panzeri4

1Institute for Adaptive and Neural Computation, School of Informatics, University of Edinburgh, Edinburgh, EH8 9AB, UK.

2Max Planck Institute for Biological Cybernetics, Spemannstrasse 38, 72076 Tuebingen, Germany.

3Division of Imaging Science and Biomedical Engineering, University of Manchester, Manchester, M13 9PT, UK.

4Italian Institute of Technology Center for Neuroscience and Cognitive Systems, Rovereto, Trento 16163, Italy.

*Correspondence: scott.lowe@ed.ac.uk

Key words: V1, visual, cortical, laminar, information, natural, current source density, spatial resolution, frequency.

\section{Running Title}
Laminar distribution of information in cortical oscillations.

\section{Summary}
Write the abstract last

\section{Highlights}
{\textless}=4 bullet points to be displayed online

\begin{itemize}
\item In V1, stimulus information is encoded into two independent frequency bands 
\item Within these bands, information is located at distinct cortical depths
\item The two frequency bands contain information about different spatial resolutions
\end{itemize}
\section[Introduction]{Introduction}
The cortical column is widely regarded as the fundamental processing unit of the neocortex (Mountcastle, 1957). Under this hypothesis, there is a common microcircuit spanning the depth of the cortex which is repeated across the cortical plane. As \ Tthe circuitry of the microcolumn is expected to have structural and functional similarities across the sensory modalities, therefore understanding this generic circuitry wouldof the columnar computation will have far reaching impacts. However, despite some progress towards understanding knowing the different types of neurons in present in different layers [CITATION], and the distribution of their most prominent inter-connections [CITATION], and aa structural wiring diagram for the cortical microcircuit [CITATION], is still unknown. Furthermore, the functional structure and computation of the microcircuit, which is to say the purpose of the processing in each cortical layer, is also still unknown. In this paper, we aim to elucidate the functional structure of the cortical layers in the primary visual cortex (V1) by examining the information contained in population activity at the various layers using Local Field Potentials (LFPs).

LLFPs are thought to reflect an integration of the membrane depolarisation in the neurons surrounding the electrode location. The LFP captures changes within the dendritic trees of neighbouring neurons as well as the soma. The lLow frequency LFP ({\textless} XXXX Hz) captures slower changes in the population activity, and reflects more of the dendritic level of processing, integrated over a larger region than the high frequency LFP (Leski, Linden, Tetzlaff, Pettersen, \& Einevoll, 2013). The idea of isolated frequency bands conveying complementarySince early EEG studies, it is has been hypothesized that different frequencies convey complementary information [CITATION] originates with EEG studies, but and this has subsequently been extended to the LFP [CITATION NEEDED]. 

We previously found that in the macaque V1 there are two LFP frequency bands, 1-8Hz and 60-100Hz, which encodecontaining independent information in the macaque V1 about natural stimuli (Belitski et al., 2008). In this study we expand the previous study by studying information as a function of cortical depth, and identify one aspect of natural scenes which is encoded differently by the two cortical frequency bands. We hypothesised the two bands of information are generated through different cortical processes and originate at different locations in the cortex. In this study we expand the previous study by studying information as a function of cortical depth, and identify one aspect of natural scenes which is encoded differently by the two cortical frequency bands.

NEEDS SUMMARY OF CURRENT LITERATURE

Recent work has shown stimulation in V1 induces gamma activity in V4 (feedforward), whilst stimulation in V4 induces alpha oscillations in V1 (feedback) (van Keroerle et al., 2014).

NEEDS SUMMARY OF OUR FINDINGS

\section{Results}
Four anesthetized monkeys (Macaca mulatta) were presented with a Hollywood movie clip, repeated 40--150 times, whilst we recorded neural activity in the primary visual cortex (V1) with a multicontact laminar electrode. Each electrode housed 16 equally spaced (150\hspace{0.2em}\textgreek{m}m) contacts spanning a total depth of 2250\hspace{0.2em}\textgreek{m}m, and was inserted perpendicular to the cortical surface (Fig. 1A). We identified the depth of each probe using the CSD, and identified granular (G), supragranular (SG), and infragranular (IG) cortical regions based on literature (see Experimental Methods for details). We performed reverse correlation between the rate of change of luminance for each pixel in the movie and the MUA at each contact site to find the spatial receptive field (RF; Fig 1B). The RF locations did not vary with depth, indicating that all electrode contacts were recording from the same cortical column.

\subsection[Distribution of information across depth and frequency]{Distribution of information across depth and frequency}
Figure 1C shows, at three cortical depths, CSD traces from eight example trials during a portion of the stimulus. The traces have been filtered within three frequency bands: 4--16Hz, 28--44Hz and 60--170Hz. One can observe that the low-frequency activity repeats across trials for the G and IG depths. Activity in the 28--44Hz range is inconsistent at all depths, and does not seem to be stimulus modulated. The envelope amplitude of the 60--170Hz band is also consistent across trials, most clearly for the SG depth. The mutual information We quantified these observations by computing the amount of information about the movie contained in the neural activity.

We computed information about which frame is currently on screen in various frequency components of the LFP and CSD (see Experimental Methods). Excluding boundary effects at the top and bottom of the cortex where white matter contaminates estimates, power is fairly smooth across depth and decays as frequency increases (Fig 2A-B). \ However, the information contained in the power does not have such a smooth distribution and differs from this in both space and frequency domains. Instead, information is contained in specific frequencies at specific depths, in a similar manner for LFP and CSD (Fig 2C--D), with prominent maxima in the 4--16Hz range at the top of the G region, and the 60-250Hz range near the top of the SG region. Additionally, there are local maxima in IG for both the 4--16Hz and 60-250Hz ranges. These results are consistent across sessions (Sup Fig 2). As the information in the CSD has better spatial localisation than the LFP (Einevoll, Kayser, Logothetis, \& Panzeri, 2013; Kajikawa \& Schroeder, 2011), for the remainder of the paper we study only the CSD.

These findings suggest that there are different information channels in a single cortical column.

\subsection{Information redundancy between frequencies}
Having identified the most informative regions in depth and frequency, there are two possibilities: either these regions contain the same information about the stimulus, through transcoding of one frequency range to another across the cortex [CITE COMPUTATIONAL STUDY ON THIS], or the regions contain different information about the stimulus. We investigated how similar the information was by computing the redundancy of information contained in pairs of frequencies (see Experimental Methods). We found there are two frequency domains within which information is redundant: 4--40Hz and {\textgreater}40Hz (Fig 3A). Furthermore, the information contained in neural frequencies {\textless}40Hz is different to the information contained in frequencies {\textgreater}40Hz, since these measured to be independent (redundancy ${\leq}$0\%). The same {\textless}40Hz and {\textgreater}40Hz division is observed for the signal correlation (Sup Fig 3), and our results corroborate earlier findings (Belitski et al., 2008). Based on these and the above results, we extracted two bands (4--16Hz and 60--170Hz) that contain the most information and independently encode information about the stimulus.

\subsection{Information redundancy across depth}
Having established the independence of these bands, we investigated the redundancy between the power of oscillations at different cortical depths (Fig 3B). For the 4-16Hz frequency range, we found there is some redundancy across the entire cortical depth, but there are two distinct cortical regions (above and below the CSD reversal, marked as 0mm depth) within which information is more redundant. These findings are in agreement with (Maier, Adams, Aura, \& Leopold, 2010), who found a transition corresponding to the G/IG boundary which isolated two cortical regions with high coherence {\textless}100Hz.

Gamma oscillations (60-170Hz) code, with substantial redundancy across the cortical depth, with some compartmentalisation of SG and IG activities. In addition we also included the MUA signal, which corresponds to the local population firing rate. There is less redundancy of information across cortical depths for MUA than for gamma; this observation is due to spiking activity being more localised than gamma oscillations. In agreement with previous findings (Belitski et al., 2008), we find that information contained in the gamma range and information in the MUA are redundant with each other. This is to be expected, since MUA activity is known to be correlated with the gamma cycle.(due to peaks/troughs in gamma relating to peaks/troughs in firing rate)

Blah, blah, blah

Overall redundancy is summarised in Fig 3C, which shows the average across all cortical depths for each pair of frequency bands.

Importantly, we find the information in the 4-16Hz range is independent of the information contained in both gamma and MUA frequency ranges across all cortical depths. In particular, this means the two localised high information regions in depth-frequency space from Figure 2D contain independent information to one-another. Importantly, this mean this argues against a situation where SG contains the same information as G/IG activity transcoded from low-frequency to high-gamma oscillations; at least some of the information is unique to each.

\subsection{Information about spatial frequency components of visual stimulus}
In the above, we have seen there are two frequency bands in V1 which, across all the cortical depth, contain independent information to each other. Next we investigate what aspects of the visual scene these two independent components contain. Since neurons in the primary visual cortex are known to respond strongly to moving sinusoidal gratings with specific spatial frequencies, we considered how much information the frequency bands contained about changes in luminance as a function of spatial frequency. Hereto, we decomposed the series of frames in the movie into set of spatial frequency components by finding the rate of change of luminance within a given set of spatial frequency bands (see Fig 4; Experimental Methods), and then computed the amount of information about this series contained in the neural activity.

We found the low frequency CSD bands ({\textless}40Hz) contained more information about low, coarse spatial frequencies (0.1--0.6cpd), whereas the higher frequencies ({\textgreater}40Hz) contained more information about high, fine spatial frequencies (0.6--5.0cpd) (Fig 5B). This was not a continuous transition; instead we observe an abrupt change at 40Hz, with lower and higher neural oscillation frequencies tuned to stimulus features with different spatial frequencies. This was true across the entire cortical depth (Fig 5C--D), where the two frequency bands (4--16Hz and 60--170Hz) contained information about opposing spatial frequencies. The distribution of information across the cortical depth corresponds to that found in Fig 2D. These results are summarised in Fig 5A, which shows the average across the cortical depth. Information reaches its maxima around 0.2cpd for the 4--16Hz frequency range and 2.5cpd for the 60--170Hz frequency range.

In Fig 6, we summarise the previous results by extracting two spatial frequency bands: coarse ({\textless}0.3cpd, low-pass spatial filter) and fine ({\textgreater}1cpd, high-pass spatial filter), example traces for which are shown above Fig 6A. These spatial components have low correlation between them (Fig 6B; r=0.18). Example CSD traces are also shown for two electrode contacts over same time period (left side). Note that peaks and troughs in the coarse luminance signal are coincident with peaks and troughs with the alpha power, and similarly for the fine luminance and gamma power, indicating the positive relationship between stimulus and cortical response. This observation is quantified by the correlation and mutual information between these components (Fig 6A).

\subsection{Layer 1 60-170Hz amplitude is coupled to Layer 5 4-16Hz phase}
In previous section ``Information redundancy across depth'', we showed that high and low LFP frequencies contain independent information to one-another. To further investigate the relationship between these two bands, we computed the cross-frequency coupling between the low frequency phase and high frequency oscillation amplitude. In agreement with previous work (Spaak, Bonnefond, Maier, Leopold, \& Jensen, 2012), we found there is significant coupling between the 4-16Hz phase in lower G and mid IG with the and amplitude of 60-170Hz oscillations in upper SG (Fig 8). There is also localised coupling between the 4-16Hz phase with 60-170Hz amplitude in G and IG. These findings were all true of both the stimulus driven and spontaneous recordings.

\section{Discussion}
In summary, we find while LFP power is smooth and its depth profile is close to flat (Fig. 2a,b) the information that the LFP encodes reveals much more structure. We found there are two cortical regions at which oscillations in these frequency ranges are much more informative. Namely 4--16Hz at upper granular and mid-infragranular, and 60-170Hz at upper supragranular and mid{}-infragranular regions.Previous work (Belitski et al., 2008) has shown that in the macaque primary visual cortex information is coded in two frequency bands ({\textless}40Hz and {\textgreater}40Hz) containing independent information about natural visual scenes. Our analysis extended across the cortical depth, and we found there are two cortical regions at which oscillations in these frequency ranges are much more informative (4--16Hz at upper G and mid- IG; 60-170Hz at upper SG and mid-IG regions).

Moreover, 

We also examined whether changes in luminance at different spatial frequencies induced differential changes in the cortex as a function of neural frequency and depth. Namely, high spatial frequencies are encoded in oscillations faster than 40Hz and low spatial frequencies are encoded in oscillations slower than 40Hz. We found that frequencies below and above 40Hz contain information about different spatial frequencies.

There are multiple possible interpretations of these findings, of which one, many, or even none may be correct. Firstly, it is conceivable that the coding of different aspects of the stimulus into different frequency bands is a computational strategy of the cortex. Our results suggest there is multiplexing in the cortex, with low frequency and high frequency oscillations of the same population activity simultaneously encoding low and high spatial frequency components of the stimulus respectively. The idea of different frequency bands conveying different spatial frequency components of the stimulus has been proposed before from the results of an EEG study (Smith, Gosselin, \& Schyns, 2006).

\ one would expect that if certain oscillation frequencies in the visual cortex contain information about specific aspects of the stimulus, this is likely to be because the brain has encoded this information into oscillations in the activity of the local population. This would only make sense if the information is utilized by the brain in order to interpret its stimuli. Consequently, our results indicate there is multiplexing in the cortex, with low frequency and high frequency oscillations of the same population activity simultaneously encoding low and high spatial frequency components of the stimulus respectively. Intuitively, information contained in the two frequency bands can be combined by downstream visual cortical regions to regain the original stimulus as necessary. \ The idea of different frequency bands conveying different spatial frequency components of the stimulus has been proposed before from the results of an EEG study (Smith, Gosselin, \& Schyns, 2006).

Additionally, we can speculate about why separating the visual scene into low frequency (coarse) and high frequency (fine) components in V1 is useful. One possibility is that low frequency oscillations are output from V1 along the dorsal visual stream, whereas high frequency oscillations travel propagate through the ventral stream [IS THERE EVIDENCE FOR THIS?]. Another possibility is that broad, coarse changes in the stimulus are useful for making rapid responses in the motor cortex to sudden changes, such as approaching threats.

Separation into low and high frequency domains with different properties seems to be a common property of the cortex. In motor cortex, activity at {\textless}13Hz and {\textgreater}60Hz relates to behaviour but there is a separating band \~{}30Hz which does not (Rickert et al., 2005). In the hippocampus, there is a gating effect between 30Hz and 40Hz, with lower but not higher frequencies able to propagate to the cortex (Moreno, Morris, \& Canals, 2015). [also some unpublished research by Julian Hoffman into independent oscillations in the barrel cortex]. This suggests this encoding scheme is common across the cortex. Some studies have suggested that the coupling of oscillations between two cortical regions facilitates the transmission between them [CITE SOME EXPERIMENTAL \& COMPUTATIONAL WORKS].

A separation of visual stimuli into coarse and fine channels is known to occur before the stimuli arrive in the cortex. The outputs from different types of retinal ganglion cells (RGCs) travel to the cortex through different regions of the LGN. The M-pathway arises from RGCs with large, achromatic receptive fields, and projects mainly onto layer 4C\textgreek{a }in V1. The P-pathway originates with RGCs with smaller, chromatic receptive fields providing higher spatial resolution but lower temporal resolution; this pathway projects onto layer 4C\textgreek{b (}E M Callaway, 1998). It is possible that the two frequency channels in V1 relate to the two pathways providing its inputs.. [Cite Nathaniel J Killian from AREADNE on information in low-frequencies of LGN. Nothing available to actually cite?]

Since layer 4 is generally regarded as the primary layer of V1 which receives afferent inputs from the LGN, some readers might wonder how information in the gamma band has ``arisen'' in SG layers without passing through G. However, our results do not necessitate this. Fine-resolution information about the visual stimulus can arrive from the LGN into L4 of V1, with the information encoded into which neurons the afferent connections target. This information is not detectable from the population level activity. 

As many readers will be aware, it has long been known that neurons in the primary visual cortex have a response curve tuned to a preferred spatial frequency. Work demonstrating the spatial frequency preference of single neurons typically involves the presentation of moving sinusoidal grating with a particular spatial frequency. [NOT SURE WHERE I WAS GOING WITH THIS{\dots}]

The higher frequency band contains information about higher spatial frequencies changes in the stimulus. This corresponds to the detection of edges and texture, which are properties that single neurons in V1 are known to be selective for. 

We observed that each frequency has a similar amount of power across the cortical depth, but oscillations at these frequency ranges contain much more information at particular cortical depths. This is curious as it indicates that, for any given frequency band, oscillations are present in all cortical depths, but most of the oscillations exhibited are not stimulus encoding. This seems wasteful. [important point]

We observed qualitatively that information-carrying events in layer 4 were large, temporary deflections with a long duration (low frequency), whereas layers 5/6 contained sustained oscillations (see Fig 1A for examples). The deflections in L4 were usually coincident with scene cuts or rapid changes in the stimulus. This could be interpreted as an error signal, since sudden, large changes in the stimulus would result in any predictive model of the stimulus making large errors. However, a more simple interpretation is these deflections correspond to changes in the afferent input to V1 from LGN. In support of this, we note that the spatial scale of the information in the low frequency band (0.25cpd) approximately corresponds to the size of receptive fields for regions of the V1 corresponding to the parafovea (2 degrees).

The sustained oscillations in L5/6 also contain information about coarse changes in the stimuli. These cortical layers are known to have connections to the motor cortex, feedback to the LGN and receiving feedback from higher cortical regions. 

Recent work has indicated that alpha and gamma bands are important for feedback and feedforward activity respectively (van Keroerle et al., 2014). This study (van Keroerle et al., 2014) found that gamma waves are initiated at layer 4 and propagate outwards to the top of SG and bottom of IG, with alpha waves propagating in the opposite direction. Our study finds the most information in gamma bands at the very top (and very bottom) of the cortex, and the most information in alpha bands at the top of L4 (and L6). Reconciling these results together, we find that there is most information in the power of the alpha and gamma oscillations at the cortical depths where they terminate, and the least where they originate. This suggests that the oscillations are generated at one cortical depth without much stimulus dependency, but as the oscillations propagate up and down the cortex they are either amplified or supressed in a stimulus dependent manner.


\bigskip

In agreement with previous work (Spaak et al., 2012), we found there was cross-frequency coupling between the stimulus-encoding power of gamma oscillations in L1 and the phase of alpha oscillations in lower L4. Anatomically, we believe this is related to the pyramidal cell bodies in L5A, which have apical dendritic tufts in L1 (Hill, Jia, Sakmann, \& Konnerth, 2013; Zhu \& Zhu, 2004). This cross-frequency coupling could be one mechanism through which the L1 gamma wave containing high levels of information about the stimulus is converted into an alpha oscillation for feedback into the hierarchically lower cortical region. Neurons in L5 are known to be related to long-range cortical output (Hill et al., 2013). 

Other references to use

(E. M. Callaway, 1998)

(Harris \& Mrsic-Flogel, 2013)

(Bannister, 2005)

(Mountcastle, 1997) 

(Victor, Purpura, Katz, \& Mao, 1994)

(Hansen, Chelaru, \& Dragoi, 2012)

(Spaak et al., 2012)

(Maier et al., 2010).

(Self, van Kerkoerle, Super, \& Roelfsema, 2013)

\section[Experimental Methods]{Experimental Methods}
\subsection{Ethics Statement}
Data was collected from the primary visual cortex (V1) of four healthy rhesus monkeys (Macaca mulatta; four males 8--11 kg; 10--12 years). All the experimental procedures were approved by the local authorities (Regierungspr\"asidium, Baden-W\"urttemberg, T\"ubingen, Germany; Project Nr. KY4/09) and were in full compliance with the guidelines of the European Community (EUVD 86/609/EEC) and were in concordance with the recommendation of the Weatherall report for the care and use of non-human primates (Weatherall, 2006). The animals were socially (group-) housed in an enriched environment, under daily veterinarian care. Weight, food and water intake were carefully monitored on a daily basis.

\subsection{Anesthesia for Neurophysiology Experiments}
The anesthesia protocol for all the experimental procedures have been described previously (Logothetis, Guggenberger, Peled, \& Pauls, 1999; Logothetis, Pauls, Augath, Trinath, \& Oeltermann, 2001). Briefly, glycopyrrolate (0.01 mg{\textperiodcentered}kg{}-1) and ketamine (15 mg{\textperiodcentered}kg{}-1), were used previous to general anesthesia. Induction with fentanyl (3 mg{\textperiodcentered}kg{}-1), thiopental (5 mg.kg{}-1) and succinylcholine chloride (3 mg.kg{}-1), animals were intubated and ventilated using a Servo Ventilator 900C (Siemens, Germany) maintaining an end-tidal CO2 of 33-35 mm Hg and oxygen saturation above 95\%. The anesthesia was maintained with remifentanil (0.5--2 \textgreek{m}g.kg{}-1min) and mivacurium chloride (2--6 mg.kg{}-1h) which ensured no eye movement during electrophygiological recordings. The anesthetics dosage were established by measuring stress hormones and were selected to ensure unaffected physiological response at normal catecholamine concentrations \ (Logothetis et al., 1999). In addition, it has been shown that using remifentanil has no significant effect on the neurovascular and neural activity of brain areas that do not belong to the pain matrix (Goense \& Logothetis, 2008; Zappe, Pfeuffer, Merkle, Logothetis, \& Goense, 2008). In particular, visual cortex does not bind remifentanil. We monitored the physiological state of the monkey continuously and kept within normal limits. Body temperature was tightly maintained at 38--39{\textdegree}C. Throughout the experiment lactate Ringer's (Jonosteril, Fresenius Kabi, Germany) with 2.5\% glucose was continuously infused at a rate of 10 ml.kg{}-1.hr{}-1 in order to maintain an adequate acid-base balance and intravascular volume and blood pressure were maintained by the administration of hydroxyethyl starch as needed (Volulyte, Fresenius Kabi, Germany). 

We used anesthetized animals because the preparation allows longer data acquisition times and to associate particular neural events to specific stimulus features without the strong effects of animal cognitive state, including effects of attention and arousal that would introduce additional complication in the interpretation of signals.

\subsection{Visual Stimulation}
A few drops of 1\% cyclopentolate hydrochloride were used in each eye to achieve mydriasis. Animals were wearing hard contact lenses (W\"ohlk-Contact-Linsen, Sch\"onkirchen, Germany) to focus the eyes on the stimulus plane. \ The visual stimulation in all experimental sessions was presented in the eye with stronger ocular preference of recording sites. The stimulus was presented using either an in-house custom-built projector (SVGA fibre-optic system with a resolution of 800x600 pixels, a frame rate of 30Hz), or a CRT monitor (Iiyama MA203DT Vision Master Pro 513, frame rate 118Hz) placed at eye level, 50 cm in front of the eye. We found the same results with both display devices, except that monitor refresh with the30Hz stimulus induced cortical oscillations at 30Hz. Since this is the result of using an artificial stimulus with a low refresh rate (a well-known issue at this stimulus frequency), we removed this from the data (see Artefact Removal) and pooled the results across all sessions. The visual stimulus consisted of high contrast (100\%), gamma corrected, fast-moving, colourful movie clips (no soundtrack) from commercially available movies. Stimulus timings were controlled by a computer running a real-time OS (QNX, Ottawa, Canada). Stimulus-on periods of 120s (5 sessions; 1 session: 40s) were interleaved with stimulus-off periods (isoluminant grey screen) of 30s. 

\subsection{Neurophysiology Data Collection and Analysis}
The electrophysiological recordings were performed by doing a small skull trepanation, after which the dura was visualized with a microscope (Zeiss Opmi MDU/S5, Germany) and carefully dissected. The electrodes were slowly advanced into the visual areas under visual and auditory guidance using manual micromanipulator (Narashige Group, Japan). Electrodes consisted of laminar probes (NeuroNexus Technologies, Ann Arbor, USA). These electrodes contained 16 contacts on a single shank 3 mm long and 150 {\textmu}m thick. The electrode sites were spaced at 150{\textmu}m apart, with a recording area of 413 {\textmu}m2 each. We used a flattened Ag wire, which was positioned under the skin, as reference electrode (Murayama et al., 2010). The recording access was filed with a mixture of 0.6\% agar dissolved in NaCl 0.9\%, pH 7.4 solution which guaranteed good electrical connection between the ground contact and the animal (Oeltermann, Augath, \& Logothetis, 2007). The impedance of the contact points was always measured during the experiments and ranged from 480 to 800 k{\textohm}. The signals were amplified and filtered into a broadband of 1Hz--8kHz (Alpha-Omega Engineering, Nazareth, Israel) and then digitized at 20.833 kHz with 16 bit resolution (PCI-6052E; National Instruments, Austin, TX).

\subsection{Luminosity Function}
In order to best approximate the luminosity perceived by macaques, we relied on analogies with the human visual system. Research with humans suggest the luminosity function is linearly related to the long (L) and medium (M) cone activation, and independent of the short (S) cone activation (Stockman, Jagle, Pirzer, \& Sharpe, 2008). Furthermore, the weighting of L and M activations towards perceived luminance is believed to be similar to the L:M ratio in the individual (Stockman et al., 2008). Old world monkeys such as macaques have an L:M ratio which is approximately 1:1 (Dobkins, Thiele, \& Albright, 2000), so we assumed a luminosity function equally weighed between the L and M cone activations,  $Y=L+M$. The 10{\textdegree} cone fundamentals of Stockman \& Sharpe (2000) were used since the cone fundamentals of old world monkeys are known to be very similar to humans (Dobkins et al., 2000). By taking the product of the emission spectra for pure red, green and blue with the luminosity function, integrating over wavelength and normalising, we obtained the following equations for relative luminance in terms of pixel intensity for the two devices used in the experiment .

 $Y_{\mathit{projector}}=0.2171{\bullet}R+0.6531{\bullet}G+0.1298{\bullet}B$ $Y_{\mathit{CRT}}=0.1487{\bullet}R+0.6822{\bullet}G+0.1691{\bullet}B$

\subsection{Artefact Removal}
An artefact removal procedure was performed to reduce the effects of line noise (one session) and monitor refresh (the three sessions with 30Hz stimulus). Artefact frequencies were identified by large, localised peaks in the power spectral density, which was computed with the periodogram method (see Supplementary Table 1). In each case, the average artefact waveform was found and subtracted from the recorded signal. To correct for phase shifts of the artefact, the averaging and subsequent subtraction were performed in blocks of 50 artefact periods with a phase chosen to maximise the cross-covariance of the signal with the artefact waveform.

\subsection[Current Source Density]{Current Source Density}
The Current Source Density (CSD) was computed using the inverse CSD method (Pettersen, Devor, Ulbert, Dale, \& Einevoll, 2006). To compute this, we used a \textgreek{d-}source model of local field generation with a diameter of 500\textgreek{m}m, chosen to correspond to the effective size of columnar activity (Horton \& Adams, 2005; Lund, Angelucci, \& Bressloff, 2003). Since this method requires an even spacing between voltage measurements, gaps caused by faulty recording contacts in the electrode were filled in with a local average (W\'ojcik \& Leski, 2010). A homogeneous cortical conductivity of 0.4 S/m was assumed (Logothetis, Kayser, \& Oeltermann, 2007). The agar solution placed on top of the recording access point had an NaCl concentration of 9 mg/ml, and the conductivity of this was estimated to be 2.2 S/m (Kandadai, Raymond, \& Shaw, 2012). The CSD was spatially smoothed with a three-point Hamming filter (add ref that justifies this smoothing).

\subsection{Multiunit Activity }
Figure 1: Multi-unit activity (MUA) was calculated by band-passing the voltage recording between 900 and 3000Hz with a zero-phase sixth-order Butterworth filter, taking the absolute value, applying a 300Hz low pass third-order Butterworth filter, and then downsampling. This yields a smoothed spike rate, analogous to a population firing rate.

Figure 3: Multi-unit activity (MUA) was calculated by downsampling by a factor of 3, band-passing the voltage recording between 900 and 3000Hz with a zero-phase sixth-order Butterworth filter, taking the absolute value, downsampling by a further factor of 12.

\subsection{Receptive Field Locations}
\ The spatial receptive fields (RFs) were found by reverse correlating the MUA and the pixel-by-pixel Z-scored frame-by-frame difference in luminance with a fixed lag of 66.7ms. The rate of change in luminance was used because it is known to correlate well with thalamic drive. For each session, the RF centre was located using the average of the reverse correlation across all cortical channels.

\subsection{Identification of Cortical Laminae}
Depth calibration of the electrode was performed by considering the spikes and CSD induced by both the onset of the movie and by 100ms full-screen flashes (6s flash interval). From the measured potentials, we determined the CSD and spike densities. Spikes were detected \ by high pass filtering the raw signal above 500Hz with a zero-phase eighth-order Butterworth filter, and classifying any points more than 3.5 standard deviations above the mean signal during pre- and post-stimulus periods as a spike, with a minimum inter-spike-interval of 1ms. The majority of thalamic afferents in V1 stimulate layer 4 (indirectly: see Hansen et al. (2012)), with the first cortical response manifesting at layer 4C\textgreek{a (}E. M. Callaway, 1998), resulting in an initial current sink and first burst of spiking activity located here. For each recording session, we found the contact exhibiting the first spiking and CSD response, the centre of the most responsive region, and the centre of the first CSD sink for both the movie onset and flash evoked activity. We took the average of these 8 locations and identified the closest electrode contact as the location of layer 4C\textgreek{a. }We estimated the laminar location of the rest of the recording depths by cross-referencing literature describing average thickness of cortical laminae in Macaca mulatta, area 17, (Lund, 1973). From this, the cortical depth was divided into 3 broad regions: supragranular (SG; layers 1--3), granular (G; layer 4), and infragranular (IG; layers 5--6). High-resolution MRI scans of two of the animals were used to determine their cortical thickness at the recording location (see Supplementary Materials).

\subsection{Identification of cortical laminae}
To identify the depth of each contact, we measured the potential evoked in response to the onset of the movie clip, and in response to full-screen maximum-luminance 100ms flash stimuli with a 6s interval. From the measured potentials, we identified the boundary between the granular (G; layer 4) and infragranular (IG; layers 5--6) regions as the source/sink reversal in the evoked current source density (CSD; see Experimental Procedures and Supplementary Materials). We estimated the location of the boundary between the granular and supragranular (SG; layers 1--3) regions by cross-referencing literature describing average thickness of cortical laminae in Macaca mulatta, area 17 (Lund, 1973).


\bigskip

\subsection[Power as a function of depth and frequency]{Power as a function of depth and frequency}
To compute power and information as a function of temporal frequency, the cortical data (LFP and CSD) were filtered in a series of bands each with a fractional bandwidth of 50\%, because cortical power falls off rapidly with frequency in a 1/f relationship. Each successive band begins and ends with frequencies 1.291 times higher than the last, so that each band has 0\% overlap with bands further away than its immediate neighbours and a 44\% and 56\% overlap with its preceding and succeeding bands respectively. The data was filtered with a zero-phase sixth-order Butterworth filter, after which the instantaneous power was estimated by taking the squared absolute value of the Hilbert transform. The power in each band was integrated over a series of 50ms windows, centred at the time of each frame change in the movie (once every 33ms, leading to a 50\% overlap of neighbouring windows). \ The power in the 4--16Hz and 60--170Hz bands was computed similarly. Fig 3A--B are plotted with power values averaged over all windows and trials, then expressed in decibels relative to the average power 1.5--248Hz (estimated by summing the power in alternate bands). Throughout Figures 3 and S3, datapoints are shown at the band centres.

\subsection{Information as a function of depth and frequency}
Power in each band was computed as above, then for each frequency band and depth we took a 10-bin histogram of the power across all the 50ms windows for all repetitions. The bin edges were chosen such that 10\% of the distribution fell into each bin, and the identity of which bin the window was allocated into was taken to be its ``stimulus''. We found the mutual information between the response and which frame was on screen at that time --- the ``stimulus'' --- by computing the Shannon information using the information breakdown toolbox (Magri, Whittingstall, Singh, Logothetis, \& Panzeri, 2009). Bias due to undersampling was corrected for using the Panzeri-Treves method (Treves \& Panzeri, 1995). Each information calculation was also bootstrapped 20 times with a randomly shuffled mapping of stimulus to response (also bias-corrected). To ensure the amount of information was statistically significant, we checked each information estimate exceeded the bootstrap mean by more than 3 standard deviations of the bootstrap values. The bootstrap mean was then subtracted from the estimated information, to counter any residual bias.

\subsection[Cortical Distribution of Power]{Cortical Distribution of Power}
For each session, the distribution of power across the cortical depth (2A--B right-hand insets) was determined by normalising the power at each depth by the summed power across all cortical depths for that band. We then took an average across sessions, weighted by the number of cortical recording sites in each session to prevent faulty (omitted) electrode contact sites from distorting the result.

\subsection{Information Redundancy }
Information redundancy was computed with the same stimuli windows as used in the information calculations. Let  $S$ denote the set of stimuli, and let  $X$ and  $Y$ each be the set of powers during each stimulus in one of the frequency bands at a particular depth. \ The information in each  $I\left(X;S\right)$ and  $I(Y;S)$ was computed in the same manner as above. \ The information in the joint distribution  $I(\left\{X,Y\right\};S)$ was computed by considering each combination of the binned  $X$ and  $Y$ as a different response, yielding a total of 100 different responses for  $\{X,Y\}$.

The relative redundancy is then defined as

\begin{equation*}
\mathit{Redundancy}=\frac{I\left(X;S\right)+I\left(Y;S\right)-I\left(\left\{X,Y\right\};S\right)}{I\left(\left\{X,Y\right\};S\right)}
\end{equation*}
and was computed using the information breakdown toolbox (Magri et al., 2009). 

\subsection{Information at about Spatial Components}
The method to find the change in luminance in each spatial frequency band is illustrated in Fig 4. First, we took the 2D fast-Fourier transform of a 224px square from the movie. A fourth-order Butterworth filter with a width of one octave was applied using a mask in the Fourier domain, and the result was projected back to the spatial domain. We then took the pixel-wise difference between each spatially filtered pair of consecutive frames. To provide a measure of the amount of change in luminance at this spatial resolution, we took the absolute amount of change in each pixel and summed this within a 2 degree diameter circular window centred at the receptive field location.

Applying this to the entire movie provided a temporal sequence of luminance changes in each spatial range. Similar to before, we took a 10-bin histogram and took the identity the bin in which each luminance change fell to be the ``stimulus''. The mutual information between this stimulus and the neural response --- the power within 4--16Hz and 60--170Hz frequency bands --- was computed with a 67ms lag between stimulus and response.

\subsection{Information about Fine and Coarse Luminance Changes}
Coarse and fine luminance changes in the stimulus were isolated in the same manner as the spatial components above, but using a low-pass ({\textless}0.3cpd) and high-pass ({\textgreater}1cpd) fourth-order Butterworth filter respectively. For both the 4-16Hz and 60-170Hz CSD powers, we computed the correlation and mutual information with the coarse and fine luminance changes, and averaged across sessions.

\subsection[Information lag between granular and infragranular regions]{Information lag between granular and infragranular regions}
The information about fine and coarse stimuli contained in 4--16Hz and 60--170Hz neural frequency bands was computed as a function of the lag between stimulus and response, in steps of 1.73ms. For each cortical recording depth, we found response lag at which the information was at its maximum. For each session, the response peak-lag was averaged across the three electrode contacts in IG and also averaged across the three electrodes in G. A paired t-test was performed across all 6 sessions to test whether the G information peaked with a shorter lag than the IG information.

\subsection[Cross{}-Frequency Phase{}-Amplitude Coupling]{Cross-Frequency Phase-Amplitude Coupling}
Strength of cross-frequency coupling was measured using the Modulation Index (Tort, Komorowski, Eichenbaum, \& Kopell, 2010). CSD data was filtered for two bands, 4--16Hz and 60--170Hz, using a zero-phase sixth-order Butterworth filter, and the instantaneous phase of 4--16Hz and envelope amplitude of 60--170Hz were each estimated using a Hilbert transform. We took a histogram of the 4--16Hz phase datapoints with 16 bins each of width \textgreek{p/8 }radians, and took the average of the 60--170Hz amplitudes simultaneous with the phase datapoints in each bin. This provides a distribution of amplitude at one depth as a function of phase at another. The Modulation Index is then the normalised Kullback-Leibler divergence of this distribution from a uniform distribution.

\section{Acknowledgments}
We are indebted to Cesare Magri for his contribution to the initial part of this research. 

\ [Stefano's grant details]

This work was supported in part by the Max Planck Society, and in part by grants EP/F500385/1 and BB/F529254/1 for the University of Edinburgh School of Informatics Doctoral Training Centre in Neuroinformatics and Computational Neuroscience from the UK Engineering and Physical Sciences Research Council (EPSRC), UK Biotechnology and Biological Sciences Research Council (BBSRC), and the UK Medical Research Council (MRC).

\section{References}
Bannister, A. P. (2005). Inter- and intra-laminar connections of pyramidal cells in the neocortex. Neuroscience Research, 53(2), 95-103.

Belitski, A., Gretton, A., Magri, C., Murayama, Y., Montemurro, M. A., Logothetis, N. K., \& Panzeri, S. (2008). Low-frequency local field potentials and spikes in primary visual cortex convey independent visual information. J Neurosci, 28(22), 5696-5709.

Callaway, E. M. (1998). Local circuits in primary visual cortex of the macaque monkey. Annu Rev Neurosci, 21, 47-74.

Callaway, E. M. (1998). Local circuits in primary visual cortex of the macaque monkey. Annual Review of Neuroscience, 21, 47-74.

Dobkins, K. R., Thiele, A., \& Albright, T. D. (2000). Comparison of red-green equiluminance points in humans and macaques: evidence for different L:M cone ratios between species. Optical Society of America, 17(3), 545-556.

Einevoll, G. T., Kayser, C., Logothetis, N. K., \& Panzeri, S. (2013). Modelling and analysis of local field potentials for studying the function of cortical circuits. Nat Rev Neurosci, 14(11), 770-785.

Goense, J. B., \& Logothetis, N. K. (2008). Neurophysiology of the BOLD fMRI signal in awake monkeys. Curr Biol, 18(9), 631-640.

Hansen, B. J., Chelaru, M. I., \& Dragoi, V. (2012). Correlated variability in laminar cortical circuits. Neuron, 76(3), 590-602.

Harris, K. D., \& Mrsic-Flogel, T. D. (2013). Cortical connectivity and sensory coding. Nature, 503(7474), 51-58.

Hill, D. N. V., Z., Jia, H., Sakmann, B., \& Konnerth, A. (2013). Multibranch activity in basal and tuft dendrites during firing of layer 5 cortical neurons in vivo. PNAS, 110(3).

Horton, J. C., \& Adams, D. L. (2005). The cortical column: a structure without a function. Philos Trans R Soc Lond B Biol Sci, 360(1456), 837-862.

Kajikawa, Y., \& Schroeder, C. E. (2011). How local is the local field potential? Neuron, 72(5), 847-858.

Kandadai, M. A., Raymond, J. L., \& Shaw, G. J. (2012). Comparison of electrical conductivities of various brain phantom gels: Developing a 'Brain Gel Model'. Mater Sci Eng C Mater Biol Appl, 32(8), 2664-2667.

Leski, S., Linden, H., Tetzlaff, T., Pettersen, K. H., \& Einevoll, G. T. (2013). Frequency dependence of signal power and spatial reach of the local field potential. PLoS Comput Biol, 9(7), e1003137.

Logothetis, N. K., Guggenberger, H., Peled, S., \& Pauls, J. (1999). Functional imaging of the monkey brain. Nat Neurosci, 2(6), 555-562.

Logothetis, N. K., Kayser, C., \& Oeltermann, A. (2007). In vivo measurement of cortical impedance spectrum in monkeys: implications for signal propagation. Neuron, 55(5), 809-823.

Logothetis, N. K., Pauls, J., Augath, M., Trinath, T., \& Oeltermann, A. (2001). Neurophysiological investigation of the basis of the fMRI signal. Nature, 412(6843), 150-157.

Lund, J. S. (1973). Organization of neurons in the visual cortex, area 17, of the monkey (Macaca mulatta). J Comput Neurosci, 147, 455-496.

Lund, J. S., Angelucci, A., \& Bressloff, P. C. (2003). Anatomical Substrates for Functional Columns in Macaque Monkey Primary Visual Cortex. Cereb Cortex, 12, 15-24.

Magri, C., Whittingstall, K., Singh, V., Logothetis, N. K., \& Panzeri, S. (2009). A toolbox for the fast information analysis of multiple-site LFP, EEG and spike train recordings. BMC Neurosci, 10, 81.

Maier, A., Adams, G. K., Aura, C., \& Leopold, D. A. (2010). Distinct superficial and deep laminar domains of activity in the visual cortex during rest and stimulation. Front Syst Neurosci, 4.

Mountcastle, V. (1957). Modality and topographic properties of single neurons of cat's somatic sensory cortex. J Neurophysiol.

Mountcastle, V. (1997). The columnar organization of the neocortex. Brain, 120, 701-722.

Murayama, Y., Biessmann, F., Meinecke, F. C., Muller, K. R., Augath, M., Oeltermann, A., \& Logothetis, N. K. (2010). Relationship between neural and hemodynamic signals during spontaneous activity studied with temporal kernel CCA. Magn Reson Imaging, 28(8), 1095-1103.

Oeltermann, A., Augath, M. A., \& Logothetis, N. K. (2007). Simultaneous recording of neuronal signals and functional NMR imaging. Magn Reson Imaging, 25(6), 760-774.

Pettersen, K. H., Devor, A., Ulbert, I., Dale, A. M., \& Einevoll, G. T. (2006). Current-source density estimation based on inversion of electrostatic forward solution: effects of finite extent of neuronal activity and conductivity discontinuities. J Neurosci Methods, 154(1-2), 116-133.

Rickert, J., Oliveira, S. C., Vaadia, E., Aertsen, A., Rotter, S., \& Mehring, C. (2005). Encoding of movement direction in different frequency ranges of motor cortical local field potentials. J Neurosci, 25(39), 8815-8824.

Self, M. W., van Kerkoerle, T., Super, H., \& Roelfsema, P. R. (2013). Distinct Roles of the Cortical Layers of Area V1 in Figure-Ground Segregation. Current Biology, 23(21), 2121-2129.

Smith, M. L., Gosselin, F., \& Schyns, P. G. (2006). Perceptual moments of conscious visual experience inferred from oscillatory brain activity. Proc Natl Acad Sci U S A, 103(14), 5626-5631.

Spaak, E., Bonnefond, M., Maier, A., Leopold, D. A., \& Jensen, O. (2012). Layer-specific entrainment of gamma-band neural activity by the alpha rhythm in monkey visual cortex. Curr Biol, 22(24), 2313-2318.

Stockman, A., Jagle, H., Pirzer, M., \& Sharpe, L. T. (2008). The dependence of luminous efficiency on chromatic adaptation. J Vis, 8(16), 1 1-26.

Tort, A. B., Komorowski, R., Eichenbaum, H., \& Kopell, N. (2010). Measuring phase-amplitude coupling between neuronal oscillations of different frequencies. J Neurophysiol, 104(2), 1195-1210.

Treves, A., \& Panzeri, S. (1995). The Upward Bias in Measures of Information Derived from Limited Data Samples. Neural Comp, 7, 399-407.

van Keroerle, T., Self, M. W., Dagnino, B., Gariel-Mathis, M.-A., Poort, J., van der Togt, C., \& Roelfsema, P. R. (2014). Alpha and gamma oscillations characterize feedback and feedforward processing in monkey visual cortex. Proceedings of the National Academy of Sciences, 111(40).

Victor, J. D., Purpura, K., Katz, E., \& Mao, B. (1994). Population encoding of spatial frequency, orientation, and color in macaque V1. J Neurophysiol, 72(5), 2151-2166.

Weatherall, D. (2006). The Weatherall report on the use of non-human primates in research. London:The Royal Society.

W\'ojcik, D. K., \& Leski, S. (2010). Current source density reconstruction from incomplete data. Neural Comp, 22, 48-60.

Zappe, A. C., Pfeuffer, J., Merkle, H., Logothetis, N. K., \& Goense, J. B. (2008). The effect of labeling parameters on perfusion-based fMRI in nonhuman primates. J Cereb Blood Flow Metab, 28(3), 640-652.

Zhu, Y., \& Zhu, J. J. (2004). Rapid arrival and integration of ascending sensory information in layer 1 nonpyramidal neurons and tuft dendrites of layer 5 pyramidal neurons of the neocortex. J Neurosci, 24(6), 1272-1279.

\section[\ Supplementary Materials]{\ Supplementary Materials}

\bigskip

\begin{flushleft}
\tablefirsthead{}
\tablehead{}
\tabletail{}
\tablelasttail{}
\begin{supertabular}{m{1.6789999cm}|m{1.888cm}|m{2.3009999cm}|m{3.5479999cm}|m{5.303cm}}
Session &
Display &
Video frame rate (fps) &
Artefact Frequencies Removed (Hz) &
Size of 1 degree square in visual field (in raw video pixels)\\\hline
H05391 &
Projector &
\raggedleft 30.015 &
30 &
16.0 x 16.0 px\\
H05nm7 &
Projector &
\raggedleft 30.015 &
30, 60 &
21.3 x 21.3 px\\
H05nm9 &
CRT &
\raggedleft 118.098 &
~
 &
17.8 x 17.9 px\\
E07nm1 &
CRT &
\raggedleft 118.098 &
50, 150 &
17.9 x 17.8 px\\
F10nm1 &
Projector &
\raggedleft 30.015 &
30, 60 &
21.3 x 21.3 px\\
J10nm1 &
CRT &
\raggedleft 118.098 &
~
 &
17.8 x 17.9 px\\
\end{supertabular}
\end{flushleft}

\bigskip

The MRI part lacks methods and results. Could you please ask Daniel if he could provide this? He should have the text ready from previous papers.
\end{document}
