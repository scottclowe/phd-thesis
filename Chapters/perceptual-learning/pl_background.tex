%------------------------------------------------------------------------------
% Background to Perceptual Learning Chapter
%------------------------------------------------------------------------------
\section{Background}
\label{sec:pl_bg}
\label{sec:bgpl}

When an individual repeatedly performs a sensory perception task they will, over time, demonstrate an improvement in performance.
If the task is repeated --- frequently and over the course of several weeks --- until performance finally saturates, the effect can persist for months.
This phenomenon is known as perceptual learning, and its duration sets it apart from shorter term effects such as sensitization (a transient increase in sensitivity following a period of stimulation) and priming (a change in perception of one stimulus immediately following a different, but related, stimulus).

For the purposes of studying perceptual learning, fine-grained discrimination tasks are appropriate; since they are intrinsically difficult, they cannot be immediately solved and there is scope for improvement.
For instance, an example of a typical task chosen by neuroscientists when studying perceptual learning is that of discerning the difference between straight lines of very similar orientations, or the alignment offset between sets of straight lines, known as vernier acuity.
If it is trained, perceptual learning can be exhibited across seemingly all sensory modalities \citep{Gibson1955,Gilbert1994,Gilbert2001,Dinse2003}; other tasks which have been used for experiments include depth perception \citep{Westheimer1988,Fendick1983145}, somatosensory spatial resolution \citep{Pleger09102001,Godde1597}, estimation of weight, and discrimination of pitch \citep{demany1985perceptual,Carcagno2011}.
%\citep{Fine2002} - list of many visual tasks for perceptual learning

However, the improvements in sensory discrimination which are made through perceptual learning are highly specific to the task at hand.
For instance, training for vernier acuity only gives improvements for stimuli with the same orientation (\SI{\pm30}{\degree}) and spatial frequency ($\pm\nicefrac{1}{2}$ octave) \citep{Fiorentini1980,Poggio1991}, and training on line separation yields no effect when the lines are later replaced with dots \citep{Poggio1992}.
Moreover, results are specific to the retinotopic location of the stimulus, with translation through \SI{<10}{\degree} from the training spot sufficient to remove the effects (\citealp{Fiorentini1980}; \citealp{Fiorentini1981,Poggio1991,Karni1991}).
This said, some studies have found a limited amount of effect-transfer to regions in the opposite hemisphere for timing-dependent tasks \citep{Ball1987,Berardi1987}.

There is still some contention over where the physiological changes which lead to perceptual learning are situated in the brain.
Consequently, there are several competing models which attempt to explain how perceptual learning arises.
The ``early'' model hypothesises that improvements principally occur at a low level in the sensory cortex \citep{Gilbert2001,Fahle2005}.
The ``late'' model states that improvements are in the higher level cortical areas related to decision making \citep{Yu2004}.
Whilst according to the ``reverse hierarchy model'', improvements are made first in higher level decision areas, and then these are propagated down the cortical hierarchy to lower levels via top-down feedback signals if the changes at higher levels are insufficient \citep{Ahissar2004,Hochstein2002}.


Perceptual learning is thought to be connected to cortical remapping and reorganisation in response to similar stimuli \citep{Dinse2003,Pleger2003,Polley2006}.
In such experiments, the region of the cortex coding for the stimulus is seen to expand.
Some researchers in this field have suggested that perceptual learning might be the mechanism which underpins all adult plasticity in the sensory and association cortices \citep{Gilbert2001}.


Neural changes correlated with perceptual learning have been observed at many levels of the cortical hierarchy.
Studies have found changes in the orientation tuning curves of neurons in both \ac{V1} \citep{Schoups2001} and \ac{V4} \citep{Li2004,Yang2004,Raiguel2006}, however the effects are greater in \ac{V4} than in \ac{V1} \citep{Raiguel2006}, and not all studies find neural changes in \ac{V1} and \ac{V2} which relate to perceptual learning, even when the subject has demonstrated psychometric improvement in the task \citep{Ghose2002}.

Due to the specificity of perceptual learning, only neurons in the retinotopic area where the stimulus is located are affected.
When the properties of individual neurons have been observed to change during perceptual learning, their tuning curves for task-relevant features have become sharper.
Under activity-based models of neural information processing, this will provide more information about the task-relevant stimulus property if it falls on the steeper slope of the tuning curve.
Studies have also shown that the effect of perceptual learning is most pronounced on the most relevant neurons from the perspective of information conveyed \citep{Raiguel2006}.


% Perceptual learning has also been demonstrated in the inferotemporal region (IT) for face classification tasks in monkeys \citep{Sigala2002}.
% In this study, discriminatory features relevant to the task were more represented across the neural population than task irrelevant features.

% stuff about contrast sensitivity anomaly
Since all neurons in the visual system have contrast tuning to some degree, one might think a contrast discrimination task a good choice for a perceptual learning study.
However, perceptual learning has proven unreliable for such discrimination problems, possibly because contrast sensitivity is already overtrained due to its importance in low-light conditions.
Better results have sometimes been found if the contrast test stimulus is accompanied with flanking stimuli \citep{Adini2002}, a phenomenon known as context-dependent learning, though other studies have found learning occurs at the same rate both with and without flankers \citep{Yu2004}, despite nearly identical setup between the experiments with the conflicting two results.


When studying perceptual learning with information theory, an obvious expectation is for the information contained in the population spiking activity to increase over time as perceptual learning occurs.
It is also likely that this increase will not be symmetric across the population, with some neurons adapting their responses to the training stimulus class more than others.
In line with previous experiments \citep{Raiguel2006}, I would also expect to see more of a change in information for neurons in \ac{V4} than \ac{V1}, and also a greater change in the \ac{V4} neurons which are the most informative to begin with \citep{Raiguel2006}.
In keeping with the reverse hierarchy model, learning should begin in \ac{V4} first before being propagated down to \ac{V1}, so one would expect to see distinct increases in the mutual information between the stimulus and \ac{V4} on a shorter timescale than between \ac{V1} and \ac{V4}.

Since temporal coding, in particular response latency, has been found to be important for subtle contrast differences \citep{Reich2001,Arabzadeh2006}, I hypothesise that the amount of information in the temporal coding of the spiking data will have increased above and beyond any increase in the information contained in the firing rates alone.
Furthermore, I expect to see that response latencies become more stimulus dependent, conveying an increasing amount of information about the stimulus contrast.

Additionally, since these studies \citep{Reich2001,Arabzadeh2006} also found the information contained within firing rate alone was sufficient for gross discrimination of contrast, I hypothesise that information in the latency and temporal code will only increase significantly for test stimuli close in contrast to the sample stimulus (see \autoref{ch:exp} for an explanation of the experimental setup).
