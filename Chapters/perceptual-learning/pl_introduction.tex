%------------------------------------------------------------------------------
% Introduction
% Overall introduction to Perceptual Learning Chapter
%------------------------------------------------------------------------------
In this chapter, we investigate the neural correlates of perceptual learning within the visual cortical regions \ac{V1} and \ac{V4}.

Perceptual learning is the phenomena in which an individual becomes more adept at fine-grain discrimination of stimuli through repetitive stimulation by the stimulus class in question.
Clearly, such changes in perceptual ability are mediated by changes within the brain, but it is not currently known which changes drive the increase of such perceptual abilities.

Moreover, a long-standing question within visual perceptual learning has been whether cortical changes are driven through bottom-up or top-down developments.

Under the bottom-up hypothesis, repetitive stimulation of similar stimuli causes \ac{V1} to change its self-organisation such its representations of these stimuli are more prominent.
There is then less loss of this fine-detail information in the first layers of cortical processing, so better quality information about the task at hand can be offered up to the higher-level cortical regions which then make the classification decision with regards to each stimulus.

With the top-down hypothesis, demand for better classification performance from high-level (output) cortical regions triggers an increase in cortical feedback, and the release of neurotransmitters such as \ac{ACh}, dopamine, or norepinephrine which are all associated with increasing in the rate of change in synaptic connection strengths within the cortical region where they are present.

Using multi-unit spiking data recorded from macaque \ac{V1} and \ac{V4}, recorded by Xing Chen within the lab of Alex Thiele, Newcastle University, I investigated these hypotheses by decoding the information about the sensory stimulus encoded in \ac{V1} and \ac{V4} and comparing the rate of change of this over the course of experimental training.
