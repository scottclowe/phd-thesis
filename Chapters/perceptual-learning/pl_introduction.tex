%------------------------------------------------------------------------------
% Introduction
% Overall introduction to Perceptual Learning Chapter
%------------------------------------------------------------------------------
In this chapter, we investigate the neural correlates of perceptual learning within two visual cortical regions, the \acf{V1} and the \aclu{V4}.

Perceptual learning is the phenomena in which an individual becomes more adept at fine-grain discrimination of stimuli through repetitive stimulation with the particular stimulus class.
Clearly, such changes in perceptual ability are mediated by changes within the brain, but it is not currently known which neural changes drive the increase of such perceptual abilities.

A long-standing question within the field of perceptual learning has been whether cortical changes are driven through bottom-up or top-down developments.
Under the bottom-up hypothesis, repetitive stimulation of similar stimuli causes \ac{V1} to change its self-organisation such that its representations of these stimuli are more prominent.
This change within \ac{V1}, simply from increased exposure to the stimulus class, will naturally result in a more accurate encoding of the properties of the stimulus salient to the task.
Since the higher-level cortical regions will have better information available to them from which to make their classification decisions, their performance will increase.

With the top-down hypothesis, demand for better classification performance from high-level (output) cortical regions triggers an increase in cortical feedback, and the release of neurotransmitters such as \ac{ACh}, dopamine, or norepinephrine in multiple cortical regions, including primary sensory regions.
These neurotransmitters are associated with increasing in the rate of change in synaptic connection strengths within the cortical region where they are present.
These effect of these events stimulated by the higher-level cortical regions then cause a subsequent change in the lower and even sensory cortical regions, with neurotransmitters accelerating the rate of change of synaptic connections and feedback potentially directing the network to strengthen particular connections.

Using multi-unit spiking data recorded from macaque \ac{V1} and \ac{V4}, recorded by Xing Chen within the lab of Alex Thiele, Newcastle University, I investigated these hypotheses by decoding the information about the sensory stimulus encoded in \ac{V1} and \ac{V4} and comparing the rate of change of this over the course of experimental training.
