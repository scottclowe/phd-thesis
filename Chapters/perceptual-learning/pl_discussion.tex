%------------------------------------------------------------------------------
\section{Discussion and Conclusions}

In this chapter, we used information theoretic techniques to evaluate how the neural activity in \ac{V1} and \ac{V4} changed during repeated training on a visual domain-specific classification task.
Over the course of the training regime, the subject's ability to discriminate whether the Gabor and sinusoidal grating presented had higher or lower contrast than a \SI{30}{\percent} contrast sample stimulus improved.
We were interested in studying the neural correlates of this phenomenon, referred to as perceptual learning.
The experimental process was performed for two macaques (\ac{M1} and \ac{M2}), and we analysed the amount of information encoded in the spike trains elicited in response to the test stimulus, whose contrast was selected within a range of \SIrange{5}{90}{\percent}.

We decomposed the information about the stimulus contained in the neural activity into task-pertinent information, which helps an observer distinguish whether the presented stimulus had a contrast higher or lower than \SI{30}{\percent}, and task-nonpertinent information which only distinguish which of the \num{7} stimuli in each of the two categories was more likely.
We found the amount of information which was not pertinent to the experimental task remained the same throughout training, whereas the amount of information which was pertinent to the task increased (although this increase was statistically significant for \ac{M2} but not \ac{M1}).
These observations are compatible with the hypothesis that the cortex is rewiring itself with training in a way which is directed towards optimising the target objective provided by the experimental protocol.
It also suggests that the neurons in the visual cortex are restricted to encoded information in a certain manner, such that they can not increase the task-pertinent information at the expense of the information encoded about the stimulus which is not relevant to the behavioural task.
% One possible explanation for this is a sharpening of the contrast tuning curves for the cortical neurons, but that the tuning curves are constrained so they cannot mimic the step-function

Within \ac{V1}, the most informative neural activity was the transient response to the onset of the stimulus, an observation supported by previous literature \citep{Muller2001}.
This is the first cortical response to the stimulus after it is presented, occurring with a latency of approximately \SI{50}{\milli\second}, in which the firing rate increases sharply, but briefly.
More information can be obtained by observing the neural activity during a short slice of only \SI{10}{\milli\second} than the overall firing rate for the entire \SI{530}{\milli\second} stimulus presentation period, provided the timing of the \SI{10}{\milli\second} window is chosen appropriately.
Furthermore, the most informative part of the stimulus onset response is the beginning.
This is most likely because higher contrast stimuli elicit spikes sooner within the retina, and as a consequence the cortical response for higher contrast stimuli has lower latency.
However, the amount of information encoded in the stimulus-onset response did not increase with training.
In fact, for both subjects, it \textemph{declined} with training.
This is most likely explained by a decline in quality of the recording electrodes --- in \autoref{sec:pl_dprime} we demonstrated the sensitivity of the electrodes in \ac{V1} declined over time.

That said, the overall firing rate for the whole stimulus presentation increased in information content with training for \ac{M2}.
This was due to an increase in information in the late stages of stimulus presentation --- the final \SI{200}{\milli\second}.
Since the neural activity after the stimulus was removed contained more information about the behavioural response than the target label of the stimulus, and when the decoder trained on data from later sessions showed a significant correlation with the behavioural responses, we believe this information is indicative of latent representation of the stimulus feedback sustained after the removal of the stimulus through feedback from higher cortical regions.

As evidenced by our results with the linear decoder, \ac{V4} activity during the stimulus presentation is indicative of the behavioural response of the subject.
We trained the linear decoder to classify the group of the stimulus, giving us a reflection of the information about the stimulus contained in the cortical activity.
We did not train the decoder to predict the behavioural choices made by the subject, and yet its responses coincided with the subject's behaviour more often than expected by chance.
This phenomenon of elevated response agreement occurred after training but not before.

There was information about the both stimulus group and the behavioural response given by the subject in the sustained activity within the visual cortex after the stimulus was removed.
This increased with training for both subjects and both brain regions.

Using a decoder to classify the stimuli based on the population activity, we found that before training the subject there was more information about the stimulus in the small population of \ac{V1} neurons we recorded than in the behavioural responses of the subject.
As training progressed, the information encoded in the \ac{V1} neurons of \ac{M2} rose, but not as quickly as the behavioural performance rose.
In contrast, the \ac{V4} population co
This 

Decoding

Cite work on **choice correlations**.

We found that noise correlations falling with training did not mean the noise correlations hindered the decoder any less.
Similar results were found by \citet{Gu2011}...
% A recent paper by \citet{Gu2011} on neurons recoded from the macaque dorsal \ac{MSTd} before and after training in a head direction discrimination task, found that although pairwise noise correlations between neurons are reduced with training, this does not yield an increase in performance in a decoder.

Complex brain functions require complex tasks in order to understand their neural correlates \citep{Pitkow2017}

%------------------------------------------------------------------------------
\subsection{Decoding}
%------------------------------------------------------------------------------

Based on data from \ac{M2} \ac{V4}, one might conclude our results to offer some corroboration with those of \citet{Gu2011}, since we find a decrease in noise correlations and in increase in decoder performance.
However, comparing the performance of the decoder to a decoder based on shuffled data without the noise correlations present suggests that the improvement in decoder performance is not due to the relatively small reduction in noise correlation observed, but is from other sources.

The data from \ac{M2} \ac{V1} does not support a ``reduction in noise correlation'' hypothesis either, and indicates that neural spike rates across the \ac{V1} population recorded from are no more informative after training than they were before.
Together, results from \ac{M2} \ac{V1} and \ac{V4} suggest that information in \ac{V1} is consistent throughout training, but the ability of \ac{V4} to read out the information in \ac{V1} improves over the same period, leading to an improvement in behavioural performance.

However, these findings are not supported by the data from \ac{M1}.

% In addition to this, the increase in reponse agreement between our rate-based decoder and the animal's behavioural responses for \ac{M2} \ac{V4} (\autoref{fig:dec_j4_alla}) indicates the monkey is increasingly relying on the activity from the channels which were recorded from in \ac{V4} in order to make its decisions about the presented stimulus contrast.

For both animals, from the data from \ac{V4} we find there is a statistically significant level of agreement between decoded and behavioural trial-to-trial responses after training but not before, which shows the neural activity in \ac{V4} and the behavioural responses are correlated.
This implies that the monkey becomes dependent on the activity of the population of neurons for which the recordings are representative.
That the agreement is better without shuffling could be taken to mean the neural correlations are important in the determination of the animal's response, however shuffling does by its very nature destroy the correspondence between the trials given to the decoder and those experienced by the animal.


\subsubsection{Moved}

Since the decoder was trained on all-but-one of the trials from a single session, there is no guaranteed similarity between the decoders across sessions --- each session may end up using a different hyperplane for classification.
This corresponds to assuming that the higher-cortical areas are changing their decoding of the lower-cortical areas as time progresses.
