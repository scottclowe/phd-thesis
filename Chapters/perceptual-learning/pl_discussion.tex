%------------------------------------------------------------------------------
\section{Conclusions}

In this chapter, we used information theoretic techniques to evaluate how the neural activity in \ac{V1} and \ac{V4} changed during repeated training on a visual domain-specific classification task.
Over the course of the training regime, the subject's ability to discriminate whether the Gabor and sinusoidal grating presented had higher or lower contrast than a \SI{30}{\percent} contrast sample stimulus improved.
We were interested in studying the neural correlates of this phenomenon, referred to as perceptual learning.
The experimental process was performed for two macaques (\ac{M1} and \ac{M2}), and we analysed the amount of information encoded in the spike trains elicited in response to the test stimulus, whose contrast was selected within a range of \SI{5}{\percent} to \SI{90}{\percent}.


\subsection{Task-pertinent information}

We decomposed the information about the stimulus contained in the neural activity into task\-/pertinent information, that helps an observer distinguish whether the presented stimulus had a contrast higher or lower than \SI{30}{\percent}, and task\-/nonpertinent information, that only helps distinguish which of the \num{7} stimuli in each of the two categories was more likely.
From this, we found the amount of information which was not pertinent to the experimental task remained the same throughout training, whereas the amount of information which was pertinent to the task increased (this increase was statistically significant for \ac{M2} but not \ac{M1}).
These observations are compatible with the hypothesis that the cortex is rewiring itself with training in a way which is directed towards optimising the target objective provided by the experimental protocol, which might be provided to visual regions in the form of feedback from higher cortical regions involved with decision making.
It also suggests that the neurons in the visual cortex are restricted to encoding information in a certain manner, such that they can not increase the task\-/pertinent information at the expense of the information encoded about the stimulus which is not relevant to the behavioural task.
One possible explanation for this is that the contrast tuning curves of the cortical neurons become sharper with training, but the tuning curves are constrained such that they cannot mimic the step-function.
In a previous study analysing the same dataset, \citet{Chen2013} found that exponential functions corresponding to the psychometric performance of the subject became steeper with training, in corroboration with this idea.


\subsection{Timing of information}

Within \ac{V1}, the most informative neural activity was the transient response to the onset of the stimulus, an observation supported by previous literature \citep{Muller2001}.
This is the first cortical response to the stimulus after it is presented, occurring with a latency of approximately \SI{50}{\milli\second}, in which the firing rate increases sharply, but briefly.
More information can be obtained by observing the neural activity during a short slice of only \SI{10}{\milli\second} than the overall firing rate for the entire \SI{530}{\milli\second} stimulus presentation period, provided the timing of the \SI{10}{\milli\second} window is chosen appropriately.
Furthermore, the most informative part of the stimulus onset response is the beginning.
We previously reported \citep{Lowe2012} that splitting \SI{20}{\milli\second} windows into \num{5} bins each with duration \SI{4}{\milli\second} to capture spike timing information only yielded an increase in information above a rate code for the timing of the onset response.
This is most likely because higher contrast stimuli elicit spikes sooner within the retina, and as a consequence the cortical response for higher contrast stimuli has lower latency \citep{Albrecht2002}.
However, the amount of information encoded in the stimulus-onset response did not increase with training.
In fact, for both subjects, it \textemph{declined} with training.
This is most likely explained by a decline in quality of the recording electrodes --- in \autoref{sec:pl_dprime} we demonstrated the sensitivity of the electrodes in \ac{V1} declined over time.
The lack of improvement in the most informative \ac{V1} activity could be because the brain is not able to use this activity in the regions making the decision of how to respond behaviourally.
However, this seems unlikely as \ac{V1} is the largest cortical region and appears to play an essential part of visual processing in mammals.%
\footnote{There are direct connections from \ac{LGN} to \ac{V2} and \ac{V3}, but \ac{V1} also makes projections to both of these cortices.}
Throwing away information from by far the most informative component of the response does not seem a likely coding strategy employed by the brain, but since changes in lighting and contrast are hurdles to be overcome when identifying a stimulus it is possible that this is the case.
Usually the visual system needs to know what an object is in spite of its contrast, not to identify the contrast itself, and later stages in the visual processing hierarchy are less sensitive to changes in contrast \citep{Sclar1990}.
Because of this, it would be useful to see the results if this experiment were repeated with fine grained classification on a different stimulus property, such as orientation or spatial frequency.

Although the information in the onset-response did not increase with training, the overall firing rate for the whole stimulus presentation did rise, for \ac{M2} at least.
This was due to an increase in information in the late stages of stimulus presentation --- the final \SI{200}{\milli\second}.
Since the neural activity present after the stimulus was removed contained more information about the behavioural response than the target label of the stimulus, and the decoder trained on data from later sessions showed a significant correlation with the behavioural responses, we believe this information is indicative of latent representation of the stimulus feedback sustained after the removal of the stimulus through feedback from higher cortical regions.


\subsection{Information at the population level}

As evidenced by our results with the linear decoder, \ac{V4} activity during the stimulus presentation is indicative of the behavioural response of the subject.
We trained the linear decoder to classify the group of the stimulus, giving us a reflection of the information about the stimulus contained in the cortical activity.
We did not train the decoder to predict the behavioural choices made by the subject, and yet its responses coincided with the subject's behaviour more often than expected by chance.
This phenomenon of elevated response agreement occurred after training but not before.

There also was information about both the stimulus group and the behavioural response given by the subject in the sustained activity within the visual cortex after the stimulus was removed.
This increased with training for both subjects and both brain regions.
As discussed in \autoref{sec:pl_poststim_discuss}, this could be due to information reaching the visual cortex from the higher brain regions within the cortex associated with decision making.
Previous analysis of the same dataset found the response time of the subject fell with training \citep{Chen2013,Chen2013thesis}, which could be related to this result.

Using a decoder to classify the stimuli based on the population activity, we found that before training the subject there was more information about the stimulus in the small population of \ac{V1} neurons that we recorded than in the behavioural responses of the subject.
As training progressed, the information encoded in the \ac{V1} neurons of \ac{M2} rose, but not as quickly as the behavioural performance rose, such that after training the behavioural performance was higher than the decoder trained on \ac{V1} activity.
In contrast, the \ac{V4} population contained a similar amount of information about the stimulus as the behavioural response, and though both rose with training, this remained true throughout the experiment.
These results suggest a large amount of redundancy in the neural activity, since decision processes of the subject in principle have access to all the neurons of the brain, but perform at a level comparable with a decoder train on the activity of only around \num{20} neurons.%
\footnote{%
\label{foot:pl_decoder_saturation}%
In fact, the situation is more extreme than this.
We used greedy feature selection to investigate the performance of the decoder as a function of the number of channels available to it, and found the decoder performance saturated with only \num{8} recording channels (not shown).
}
However this is not so surprising, since it has long been known that single neurons can convey a large fraction of the information present in the behavioural response \citep{Britten1992}.
The information contained in a pooled set of neural responses saturates quickly as the size of the pool grows due to the correlations of the responses within the population \citep{Zohary1994}.
But further to this, the performance we could attain with only a handful of \ac{V1} neurons%
\footnote{
The decoder trained on \ac{V1} activity also saturated with the \num{8} best recording channels (see \hyperref[foot:pl_decoder_saturation]{Footnote~\ref{foot:pl_decoder_saturation}}).
}
was \textemph{higher} than the initial performance of the individual.
This indicates the information needed to complete the task is available before training begins, but that neural pathways  must be rewired for such information to propagate to the higher cortical regions which decide what behavioural response to provide.

By shuffling responses from recording channels across trials, we measured the impact of noise correlations on the decoder trained on either \ac{V1} or \ac{V4} activity.
We found that the impact of noise correlations on the population-level information did not fall with training, even when the pairwise noise correlations declined over the same period.
However, we note that this interpretation of the results was not obvious when we measured the accuracy of the decoder instead of the information it contained, due to the non-linear relationship between information and accuracy.
In a study of the macaque \ac{MSTd}, \citet{Gu2011} found similar results: pairwise noise correlations between neurons are reduced with training, but this does not yield an increase in performance in a decoder trained on the population activity.


\subsection{Correlations with behaviour}

% Cite work on **choice correlations**.

Previous analysis of the same dataset using \ac{AUROC} found that, on average, the probability of agreement between the spiking activity from individual recording channels in \ac{V4} and the behavioural response rose with training, and the agreement between \ac{V1} and behaviour rose for \ac{M2}, but not \ac{M1} \citep{Chen2013thesis}.
In this new work (\autoref{sec:dec-meth-prob}), we controlled for the change in behavioural performance with training, and computed the conditional mutual information between decoded population activity and behaviour (conditioning on the identity of the stimulus).

For both animals, we find that knowing the result of the decoder trained on the \ac{V4} population activity did not provide as much information about the behavioural response (beyond the information contained in the identity of the stimulus) before training began, but did yield a significant amount of information after training.
There was also an increase in information about the behavioural response contained in the activity of the \ac{V1} population for \ac{M2} (but not \ac{M1}), though the effect size was smaller than for \ac{V4}.
There are two interpretations to this result: either the subject becomes more dependent on the activity of its \ac{V1} and \ac{V4} neurons when making its decision,%
\footnote{Since we only record a small number of cortical neurons, we would here assume that the activity of the neurons which we record are representative of the cortical region as a whole.}
or that information pertaining to the subject's decision is fed back into \ac{V1} and \ac{V4} from higher cortical regions.
However, both of these interpretations are problematic.
Since we already showed that the performance of the \ac{V4} decoder and the subject's behaviour are similar throughout training, it would make more sense for the subject's decision process to be equally reliant on its \ac{V4} activity throughout training.
But similarly, there is no reason to suspect that feedback from higher cortical regions involved in the decision making process to the visual cortex should increase with training.
Furthermore, the decision of which behavioural response to provide is not necessarily finalised during the stimulus presentation period --- the subject has another \SI{400}{\milli\second} of fixation after the stimulus is removed before they are able to respond, and even then they do not necessarily respond immediately.
However, the response time does decline with training \citep{Chen2013,Chen2013thesis}, so it may be that decisions made by the subject are initially made after the stimulus is removed, but with training the subject becomes more decisive and feedback pertaining to this decision can consequently be witnessed in the visual areas during the stimulus presentation.
This seems the more likely conclusion to draw from the analysis.
In particular, we suspect that the rise in information in \ac{M1} \ac{V1} about the behavioural response is restricted to the final \SI{200}{\milli\second} of activity, which is where we see increases in information about the stimulus.
Although the \acl{V1} has long been believed to process visual information only, recent studies have shown that mouse \ac{V1} responds to locomotion, even in the dark \citep{Pakan2016,Saleem2013,Keller2012}.
This finding lends support to the idea of projections to macaque \ac{V1} from motor planning regions, which could be triggered once the subject has decided on its response to the stimulus and is planning its saccade to the response stimuli.
