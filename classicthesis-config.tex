% ****************************************************************************************************
% classicthesis-config.tex 
% formerly known as loadpackages.sty, classicthesis-ldpkg.sty, and classicthesis-preamble.sty 
% Use it at the beginning of your ClassicThesis.tex, or as a LaTeX Preamble 
% in your ClassicThesis.{tex,lyx} with % ****************************************************************************************************
% classicthesis-config.tex 
% formerly known as loadpackages.sty, classicthesis-ldpkg.sty, and classicthesis-preamble.sty 
% Use it at the beginning of your ClassicThesis.tex, or as a LaTeX Preamble 
% in your ClassicThesis.{tex,lyx} with % ****************************************************************************************************
% classicthesis-config.tex 
% formerly known as loadpackages.sty, classicthesis-ldpkg.sty, and classicthesis-preamble.sty 
% Use it at the beginning of your ClassicThesis.tex, or as a LaTeX Preamble 
% in your ClassicThesis.{tex,lyx} with % ****************************************************************************************************
% classicthesis-config.tex 
% formerly known as loadpackages.sty, classicthesis-ldpkg.sty, and classicthesis-preamble.sty 
% Use it at the beginning of your ClassicThesis.tex, or as a LaTeX Preamble 
% in your ClassicThesis.{tex,lyx} with \input{classicthesis-config}
% ****************************************************************************************************  
% If you like the classicthesis, then I would appreciate a postcard. 
% My address can be found in the file ClassicThesis.pdf. A collection 
% of the postcards I received so far is available online at 
% http://postcards.miede.de
% ****************************************************************************************************

% ****************************************************************************************************
% 1. Configure classicthesis for your needs here, e.g., remove "drafting" below 
% in order to deactivate the time-stamp on the pages
% ****************************************************************************************************
\PassOptionsToPackage{eulerchapternumbers,listings,%drafting,%
                 pdfspacing,floatperchapter,%linedheaders,%
                 subfig,beramono,parts,%eulermath,
                 dottedtoc}{classicthesis}
% ********************************************************************
% Available options for classicthesis.sty 
% (see ClassicThesis.pdf for more information):
% drafting
% parts nochapters linedheaders
% eulerchapternumbers beramono eulermath pdfspacing minionprospacing
% tocaligned dottedtoc manychapters
% listings floatperchapter subfig
% ********************************************************************

% ********************************************************************
% Triggers for this config
% ******************************************************************** 
\usepackage{ifthen}
\newboolean{enable-backrefs} % enable backrefs in the bibliography
\setboolean{enable-backrefs}{true} % true false
% ****************************************************************************************************


% ****************************************************************************************************
% 2. Personal data and user ad-hoc commands
% ****************************************************************************************************
\newcommand{\myTitle}{Decoding information from neural populations in the visual cortex\xspace}
\newcommand{\mySubtitle}{\xspace}
\newcommand{\myDegree}{Doctor of Philosophy\xspace}
\newcommand{\myName}{Scott C. Lowe\xspace}
\newcommand{\myProf}{Prof. Mark van Rossum,\space\space{University of Edinburgh}\xspace} %Institute for Adaptive and Neural Computation,
\newcommand{\myOtherProf}{Prof. Stefano Panzeri,\space\space{Istituto Italiano di Technologia}\xspace} %Center for Neuroscience and Cognitive Systems,
\newcommand{\myThirdProf}{Prof. Alex Thiele,\space\space{Newcastle University}\xspace} %Institute of Neuroscience,
\newcommand{\mySupervisor}{SUPERVISORNAME\xspace}
\newcommand{\myFaculty}{Institute for Adaptive and Neural Computation\xspace}
\newcommand{\myDepartment}{School of Informatics\xspace}
\newcommand{\myUni}{University of Edinburgh\xspace}
\newcommand{\myLocation}{Edinburgh\xspace}
\newcommand{\myTime}{\the\year\xspace}
\newcommand{\myVersion}{version 4.1\xspace}

% ********************************************************************
% Setup, finetuning, and useful commands
% ********************************************************************
\newcounter{dummy} % necessary for correct hyperlinks (to index, bib, etc.)
\newlength{\abcd} % for ab..z string length calculation
\providecommand{\mLyX}{L\kern-.1667em\lower.25em\hbox{Y}\kern-.125emX\@}
\newcommand{\ie}{i.\,e.}
\newcommand{\Ie}{I.\,e.}
\newcommand{\eg}{e.\,g.}
\newcommand{\Eg}{E.\,g.} 
\newcommand{\etal}{\textit{et al.}}
\newcommand{\NB}{{N.B.}}
% ****************************************************************************************************


% ****************************************************************************************************
% 3. Loading some handy packages
% ****************************************************************************************************
% ******************************************************************** 
% Packages with options that might require adjustments
% ******************************************************************** 
\PassOptionsToPackage{latin9}{inputenc}	% latin9 (ISO-8859-9) = latin1+"Euro sign"
 \usepackage{inputenc}				

%\PassOptionsToPackage{ngerman,american}{babel}   % change this to your language(s)
% Spanish languages need extra options in order to work with this template
%\PassOptionsToPackage{spanish,es-lcroman}{babel}
 \usepackage{babel}					

%\PassOptionsToPackage{square,numbers}{natbib}
 \usepackage[natbibapa]{apacite}
 \usepackage{natbib,natbibspacing}

\PassOptionsToPackage{fleqn}{amsmath}		% math environments and more by the AMS 
 \usepackage{amsmath}

\usepackage{textgreek}

% ******************************************************************** 
% General useful packages
% ******************************************************************** 
\PassOptionsToPackage{T1}{fontenc} % T2A for cyrillics
	\usepackage{fontenc}     
\usepackage{textcomp} % fix warning with missing font shapes
\usepackage{scrhack} % fix warnings when using KOMA with listings package          
\usepackage{xspace} % to get the spacing after macros right  
\usepackage{mparhack} % get marginpar right
\usepackage{fixltx2e} % fixes some LaTeX stuff 
\PassOptionsToPackage{smaller}{acronym} % printonlyused,
	\usepackage{acronym} % nice macros for handling all acronyms in the thesis
%\renewcommand*{\acsfont}[1]{\textssc{#1}} % for MinionPro
%
\ifcsname/bflabel\endcsname%
    % For older versions of acronym
	\renewcommand{\bflabel}[1]{{#1}\hfill} % fix the list of acronyms
\else%
	% For acronym version >=1.41
	% Prevent acronym from using bold face
	\renewcommand{\aclabelfont}[1]{\acsfont{#1}\hfill}
\fi%
%
% ****************************************************************************************************


% ****************************************************************************************************
% 4. Setup floats: tables, (sub)figures, and captions
% ****************************************************************************************************
\usepackage{tabularx} % better tables
	\setlength{\extrarowheight}{3pt} % increase table row height
\newcommand{\tableheadline}[1]{\multicolumn{1}{c}{\spacedlowsmallcaps{#1}}}
\newcommand{\myfloatalign}{\centering} % to be used with each float for alignment
\usepackage{caption}
\DeclareCaptionLabelFormat{spacedlowsmallcaps}{%
  \bothIfFirst{\spacedlowsmallcaps{#1}}{~}\spacedlowsmallcaps{#2}}
\DeclareCaptionLabelSeparator*{periodenspace}{.\enspace}
\captionsetup{%
    hypcap=true,%
    format=plain,%
    indention=0cm,%
    font={small},% ,stretch=1.05
    labelformat=spacedlowsmallcaps,%
    labelsep=periodenspace}
\usepackage{subfig}
\usepackage{rotating}

\usepackage{chngcntr}
% \counterwithin{figure}{section}

% Add padding below caption
% \setlength{\belowcaptionskip}{3pt}

% ****************************************************************************************************
% 4b. Other useful packages
% ****************************************************************************************************
% If you don't have any of these, you can find them on CTAN
%
%\usepackage{bibspacing}     % Don't need to use this, I guess
%
\usepackage{pdflscape}       % Lets you put pages into landscape
\usepackage{rotating}        % Lets you rotate tables and figures
%
\usepackage{keyval}          % Key-value decoder (Part of the graphics bundle, so probably already included)
%
\usepackage{glossaries}      % Comprehensive glossary package
%
% ********************************************************************
% Things for tables
% ********************************************************************
\usepackage{array}           % Extended version of array and table environments
\usepackage{longtable}       % For tables spanning multiple pages
\usepackage{multirow}        % Items can span multiple table rows/cols
\usepackage{dcolumn}                % Lets you align decimal points within columns of a table
\newcolumntype{d}[1]{D{.}{.}{#1}}   % Adds decimal point column type. Must specify the number of decimal places as arg.
% NB: no need to import the (essential) booktabs package, because
% classicthesis.sty does this for us
%
% ********************************************************************
% Laying out text nicely
% ********************************************************************
% Things for maths
\usepackage{amssymb,amstext} % Full math equation support
\usepackage{amsthm}          % Better theorem environments
\usepackage{amsfonts}        % For blackboard bold, etc
\usepackage{nicefrac}        % Nicer fractions
%
%\usepackage{units}           % nice units, for 10Hz with a thin space, etc
\usepackage{siunitx}         % like units, but better
\DeclareSIUnit\cpd{cpd}      % cycles per degree as a cpd unit
\DeclareSIUnit\dva{dva}      % degrees of visual angle unit
\sisetup{separate-uncertainty = true, retain-explicit-plus}
%
\usepackage{mhchem}          % For chemical formulae
%
\usepackage{soul}            % Provides hy­phen­at­able spacing
%
% For displaying text in special ways
\usepackage{verbatim}        % Useful for program listings
% NB: listings (in its own section below) is useful for program listings too.
\usepackage{framed}          % Framed and/or shaded regions
\usepackage{color}           % Set text color
\definecolor{shadecolor}{gray}{0.9} % and make a new named color
%
% ********************************************************************
% Things for including images
% ********************************************************************
\usepackage{grffile}         % allow dots in middle of filenames
\usepackage{epstopdf}        % automatically convert eps files to pdf files
% when using epstopdf, on command line you must call pdflatex with arguments:
%--shell-escape --enable-write18
% NB: we also use the svg package, but that must be done later, after including
% the classicthesis class.
%
% ********************************************************************
% Moving things around
% ********************************************************************
\usepackage{calc}            % Allows raisebox, can move baselines
\usepackage{placeins}        % Provides \FloatBarrier which is an impass for figures 
\usepackage{float}           % Improved float interface
\usepackage{setspace}        % Can set space between lines
%
% ****************************************************************************************************

% ****************************************************************************************************
% 4c. Custom commands
% ****************************************************************************************************
\newcommand{\lyxdot}{.}
\newcommand{\mm}[0]{$\mathrm{\mu m}$ }
%\newcommand{\degree}{\ensuremath{^\circ}}
\newcommand{\ih}[0]{$I_{h}$ }
\newcommand{\cm}[0]{$cm^{2}$}
\newcommand{\sq}[0]{$^{2}$ }
% ********************************************************************
% Maths things
% ********************************************************************
% Declare a font which is useful for making symbols as curly letters,
% such as \mathpzc{H}
\DeclareMathAlphabet{\mathpzc}{OT1}{pzc}{m}{it}
%%% `Log-like' maths functions
\DeclareMathOperator{\sgn}{sgn}                         % sign
\DeclareMathOperator{\E}{\mathop{\mathbb E\/}}         % expectation
\DeclareMathOperator*{\EE}{\mathlarger{\mathlarger{\mathop{\mathbb E\/}}}}   % expectation
%\newcommand{\EE}{\mathlarger{\operatorname{\mathbb E}}}
\DeclareMathOperator{\PP}{\mathbb P\/}  % probability
%
% Derivatives
\let\underdot=\d                                        % rename builtin command \d{} to \underdot{}
%\renewcommand{\d}{\operatorname{d}}                     % old method
\renewcommand{\d}{\ensuremath{\operatorname{d}\!}}      % straight operator d (\! for no space after)
%\renewcommand{\d}{\ensuremath{d}}                       % italic (variable-like) d
\newcommand{\od}[1]{\frac{\d}{\d#1}}                    % first order ordinary derivative operator
\newcommand{\odn}[2]{\frac{\d^{#2}}{\d#1^{#2}}}         % n-th order ordinary derivative operator
\newcommand{\pd}[1]{\frac{\partial}{\partial #1}}       % first order partial derivative operator
\newcommand{\pdn}[2]{\frac{\partial^{#2}}{\partial #1^{#2}}}    % n-th order partial derivative operator
%
%%% Ordinals
%\newcommand{\ord}[2]{#1#2}                              %where we are doing normal 1st, 2nd, 3rd, 4th and not n-th
\newcommand{\ord}[2]{\ensuremath{\text{#1}^\text{#2}}}  %where we are doing normal 1st, 2nd, 3rd, 4th and not n-th
%\newcommand{\nth}[1]{#1\text{-th}}                      %where the argument is mathematical and we are in math-mode
%\newcommand{\mth}[1]{$#1$-th}                           %where the argument is mathematical but we are not in math-mode
\newcommand{\nth}[1]{\ensuremath{#1\text{-th}}}         %NEW: where argument is mathematical (may be used in any mode)
\newcommand{\mth}[1]{\nth{#1}}                          %ditto. for backward compatibility.
%
%%% Vectors
\newcommand{\mtx}[1]{\left[ \begin{matrix} #1 \end{matrix} \right]} %matrix
\newcommand{\col}[1]{\mtx{#1}}                                      %column vector
\newcommand{\row}[1]{[#1]}                                          %row vector
\newcommand{\tcol}[1]{\row{#1}^T}                                   %transposed col vector
%
%%% Notation
\newcommand{\VEC}[1]{\mathbf{#1}}                                   %vector symbol typeface
\newcommand{\SET}[1]{\mathbf{#1}}                                   %set symbol typeface
\newcommand{\NSYS}[1]{\mathbb{#1}}                                  %number system typeface (R, C, Z, N)
%
\newcommand{\R}{\NSYS{R}}                                           %number system typeface (R, C, Z, N)
\newcommand{\C}{\NSYS{C}}                                           %number system typeface (R, C, Z, N)
\newcommand{\Z}{\NSYS{Z}}                                           %number system typeface (R, C, Z, N)
\newcommand{\N}{\NSYS{N}}                                           %number system typeface (R, C, Z, N)
%
%% Imaginary unit
\newcommand{\iu}{{\mathrm{i}\mkern1mu}}
%% Real and Imaginary components
\renewcommand{\Re}{\operatorname{Re}}
\renewcommand{\Im}{\operatorname{Im}}
%
% ****************************************************************************************************

% ****************************************************************************************************
% 5. Setup code listings
% ****************************************************************************************************
\usepackage{listings} 
%\lstset{emph={trueIndex,root},emphstyle=\color{BlueViolet}}%\underbar} % for special keywords
\lstset{language=[LaTeX]Tex,%C++,
    keywordstyle=\color{RoyalBlue},%\bfseries,
    basicstyle=\small\ttfamily,
    %identifierstyle=\color{NavyBlue},
    commentstyle=\color{Green}\ttfamily,
    stringstyle=\rmfamily,
    numbers=none,%left,%
    numberstyle=\scriptsize,%\tiny
    stepnumber=5,
    numbersep=8pt,
    showstringspaces=false,
    breaklines=true,
    frameround=ftff,
    frame=single,
    belowcaptionskip=.75\baselineskip
    %frame=L
} 
% ****************************************************************************************************    		   


% ****************************************************************************************************
% 6. PDFLaTeX, hyperreferences and citation backreferences
% ****************************************************************************************************
% ********************************************************************
% Using PDFLaTeX
% ********************************************************************
\PassOptionsToPackage{pdftex,hyperfootnotes=false,pdfpagelabels,backref=page}{hyperref}
	\usepackage{hyperref}  % backref linktocpage pagebackref

\pdfcompresslevel=9
\pdfadjustspacing=1 
\PassOptionsToPackage{pdftex}{graphicx}
	\usepackage{graphicx} 

% ********************************************************************
% Setup the style of the backrefs from the bibliography
% (translate the options to any language you use)
% ********************************************************************
\newcommand{\backrefnotcitedstring}{\relax}%(Not cited.)
\newcommand{\backrefcitedsinglestring}[1]{(Cited on page~#1.)}
\newcommand{\backrefcitedmultistring}[1]{(Cited on pages~#1.)}
\ifthenelse{\boolean{enable-backrefs}}%
{%
		\PassOptionsToPackage{hyperpageref}{backref}
		\usepackage{backref} % to be loaded after hyperref package 
		   \renewcommand{\backreftwosep}{ and~} % separate 2 pages
		   \renewcommand{\backreflastsep}{, and~} % separate last of longer list
		   \renewcommand*{\backref}[1]{}  % disable standard
		   \renewcommand*{\backrefalt}[4]{% detailed backref
		      \ifcase #1 %
		         \backrefnotcitedstring%
		      \or%
		         \backrefcitedsinglestring{#2}%
		      \else%
		         \backrefcitedmultistring{#2}%
		      \fi}%
}{\relax}    

% ********************************************************************
% Hyperreferences
% ********************************************************************
\definecolor{webblue}{rgb}{0,0,0.930}
\hypersetup{%
    %draft,	% = no hyperlinking at all (useful in b/w printouts)
    colorlinks=true, linktocpage=true, pdfstartpage=3, pdfstartview=FitV,%
    % uncomment the following line if you want to have black links (e.g., for printing)
    %colorlinks=false, linktocpage=false, pdfborder={0 0 0}, pdfstartpage=3, pdfstartview=FitV,% 
    breaklinks=true, pdfpagemode=UseNone, pageanchor=true, pdfpagemode=UseOutlines,%
    plainpages=false, bookmarksnumbered, bookmarksopen=true, bookmarksopenlevel=1,%
    hypertexnames=true, pdfhighlight=/O,%nesting=true,%frenchlinks,%
    urlcolor=webbrown, linkcolor=RoyalBlue, citecolor=webgreen, %pagecolor=RoyalBlue,%
    %urlcolor=Black, linkcolor=Black, citecolor=Black, %pagecolor=Black,%
    pdftitle={\myTitle},%
    pdfauthor={\textcopyright\ \myName, \myUni, \myFaculty},%
    pdfsubject={},%
    pdfkeywords={},%
    pdfcreator={pdfLaTeX},%
    pdfproducer={LaTeX with hyperref and classicthesis}%
}   

% ********************************************************************
% Setup autoreferences
% ********************************************************************
% There are some issues regarding autorefnames
% http://www.ureader.de/msg/136221647.aspx
% http://www.tex.ac.uk/cgi-bin/texfaq2html?label=latexwords
% you have to redefine the makros for the 
% language you use, e.g., american, ngerman
% (as chosen when loading babel/AtBeginDocument)
% ********************************************************************
\makeatletter
\@ifpackageloaded{babel}%
    {%
       \addto\extrasamerican{%
					\renewcommand*{\figureautorefname}{Figure}%
					\renewcommand*{\tableautorefname}{Table}%
					\renewcommand*{\partautorefname}{Part}%
					\renewcommand*{\chapterautorefname}{Chapter}%
					\renewcommand*{\sectionautorefname}{Section}%
					\renewcommand*{\subsectionautorefname}{Section}%
					\renewcommand*{\subsubsectionautorefname}{Section}% 	
				}%
       \addto\extrasngerman{% 
					\renewcommand*{\paragraphautorefname}{Absatz}%
					\renewcommand*{\subparagraphautorefname}{Unterabsatz}%
					\renewcommand*{\footnoteautorefname}{Fu\"snote}%
					\renewcommand*{\FancyVerbLineautorefname}{Zeile}%
					\renewcommand*{\theoremautorefname}{Theorem}%
					\renewcommand*{\appendixautorefname}{Anhang}%
					\renewcommand*{\equationautorefname}{Gleichung}%        
					\renewcommand*{\itemautorefname}{Punkt}%
				}%	
			% Fix to getting autorefs for subfigures right (thanks to Belinda Vogt for changing the definition)
			\providecommand{\subfigureautorefname}{\figureautorefname}%  			
    }{\relax}
\makeatother


% ****************************************************************************************************
% 7. Last calls before the bar closes
% ****************************************************************************************************
% ********************************************************************
% Development Stuff
% ********************************************************************
\listfiles
%\PassOptionsToPackage{l2tabu,orthodox,abort}{nag}
%	\usepackage{nag}
%\PassOptionsToPackage{warning, all}{onlyamsmath}
%	\usepackage{onlyamsmath}

% ********************************************************************
% Last, but not least...
% ********************************************************************
\usepackage{classicthesis} 
% ****************************************************************************************************

% ****************************************************************************************************
% 8. Further adjustments (experimental)
% ****************************************************************************************************
% ********************************************************************
% Import svg package
% ********************************************************************
% Need to include the svg package after classicthesis because it does not
% play nicely with importing several required packages too early
\usepackage{svg}             % can include svg files with \includesvg

% ********************************************************************
% Changing the text area
% ********************************************************************
% \linespread{1.3}
\linespread{1.05} % a bit more for Palatino
%\areaset[current]{312pt}{761pt} % 686 (factor 2.2) + 33 head + 42 head \the\footskip
%\setlength{\marginparwidth}{7em}%
%\setlength{\marginparsep}{2em}%

% ********************************************************************
% Using different fonts
% ********************************************************************
%\usepackage[oldstylenums]{kpfonts} % oldstyle notextcomp
%\usepackage[osf]{libertine}
%\usepackage{hfoldsty} % Computer Modern with osf
%\usepackage[light,condensed,math]{iwona}
%\renewcommand{\sfdefault}{iwona}
%\usepackage{lmodern} % <-- no osf support :-(
%\usepackage[urw-garamond]{mathdesign} <-- no osf support :-(

% ********************************************************************
% Adjust colour of acronyms
% ********************************************************************
\makeatletter
\AtBeginDocument{%
  \renewcommand*{\AC@hyperlink}[2]{%
    \begingroup
      \hypersetup{linkcolor=Black}%webbrown
      \hyperlink{#1}{#2}%
    \endgroup
  }%
}
\makeatother
% ****************************************************************************************************

% ****************************************************************************************************  
% If you like the classicthesis, then I would appreciate a postcard. 
% My address can be found in the file ClassicThesis.pdf. A collection 
% of the postcards I received so far is available online at 
% http://postcards.miede.de
% ****************************************************************************************************

% ****************************************************************************************************
% 1. Configure classicthesis for your needs here, e.g., remove "drafting" below 
% in order to deactivate the time-stamp on the pages
% ****************************************************************************************************
\PassOptionsToPackage{eulerchapternumbers,listings,%drafting,%
                 pdfspacing,floatperchapter,%linedheaders,%
                 subfig,beramono,parts,%eulermath,
                 dottedtoc}{classicthesis}
% ********************************************************************
% Available options for classicthesis.sty 
% (see ClassicThesis.pdf for more information):
% drafting
% parts nochapters linedheaders
% eulerchapternumbers beramono eulermath pdfspacing minionprospacing
% tocaligned dottedtoc manychapters
% listings floatperchapter subfig
% ********************************************************************

% ********************************************************************
% Triggers for this config
% ******************************************************************** 
\usepackage{ifthen}
\newboolean{enable-backrefs} % enable backrefs in the bibliography
\setboolean{enable-backrefs}{true} % true false
% ****************************************************************************************************


% ****************************************************************************************************
% 2. Personal data and user ad-hoc commands
% ****************************************************************************************************
\newcommand{\myTitle}{Decoding information from neural populations in the visual cortex\xspace}
\newcommand{\mySubtitle}{\xspace}
\newcommand{\myDegree}{Doctor of Philosophy\xspace}
\newcommand{\myName}{Scott C. Lowe\xspace}
\newcommand{\myProf}{Prof. Mark van Rossum,\space\space{University of Edinburgh}\xspace} %Institute for Adaptive and Neural Computation,
\newcommand{\myOtherProf}{Prof. Stefano Panzeri,\space\space{Istituto Italiano di Technologia}\xspace} %Center for Neuroscience and Cognitive Systems,
\newcommand{\myThirdProf}{Prof. Alex Thiele,\space\space{Newcastle University}\xspace} %Institute of Neuroscience,
\newcommand{\mySupervisor}{SUPERVISORNAME\xspace}
\newcommand{\myFaculty}{Institute for Adaptive and Neural Computation\xspace}
\newcommand{\myDepartment}{School of Informatics\xspace}
\newcommand{\myUni}{University of Edinburgh\xspace}
\newcommand{\myLocation}{Edinburgh\xspace}
\newcommand{\myTime}{\the\year\xspace}
\newcommand{\myVersion}{version 4.1\xspace}

% ********************************************************************
% Setup, finetuning, and useful commands
% ********************************************************************
\newcounter{dummy} % necessary for correct hyperlinks (to index, bib, etc.)
\newlength{\abcd} % for ab..z string length calculation
\providecommand{\mLyX}{L\kern-.1667em\lower.25em\hbox{Y}\kern-.125emX\@}
\newcommand{\ie}{i.\,e.}
\newcommand{\Ie}{I.\,e.}
\newcommand{\eg}{e.\,g.}
\newcommand{\Eg}{E.\,g.} 
\newcommand{\etal}{\textit{et al.}}
\newcommand{\NB}{{N.B.}}
% ****************************************************************************************************


% ****************************************************************************************************
% 3. Loading some handy packages
% ****************************************************************************************************
% ******************************************************************** 
% Packages with options that might require adjustments
% ******************************************************************** 
\PassOptionsToPackage{latin9}{inputenc}	% latin9 (ISO-8859-9) = latin1+"Euro sign"
 \usepackage{inputenc}				

%\PassOptionsToPackage{ngerman,american}{babel}   % change this to your language(s)
% Spanish languages need extra options in order to work with this template
%\PassOptionsToPackage{spanish,es-lcroman}{babel}
 \usepackage{babel}					

%\PassOptionsToPackage{square,numbers}{natbib}
 \usepackage[natbibapa]{apacite}
 \usepackage{natbib,natbibspacing}

\PassOptionsToPackage{fleqn}{amsmath}		% math environments and more by the AMS 
 \usepackage{amsmath}

\usepackage{textgreek}

% ******************************************************************** 
% General useful packages
% ******************************************************************** 
\PassOptionsToPackage{T1}{fontenc} % T2A for cyrillics
	\usepackage{fontenc}     
\usepackage{textcomp} % fix warning with missing font shapes
\usepackage{scrhack} % fix warnings when using KOMA with listings package          
\usepackage{xspace} % to get the spacing after macros right  
\usepackage{mparhack} % get marginpar right
\usepackage{fixltx2e} % fixes some LaTeX stuff 
\PassOptionsToPackage{smaller}{acronym} % printonlyused,
	\usepackage{acronym} % nice macros for handling all acronyms in the thesis
%\renewcommand*{\acsfont}[1]{\textssc{#1}} % for MinionPro
%
\ifcsname/bflabel\endcsname%
    % For older versions of acronym
	\renewcommand{\bflabel}[1]{{#1}\hfill} % fix the list of acronyms
\else%
	% For acronym version >=1.41
	% Prevent acronym from using bold face
	\renewcommand{\aclabelfont}[1]{\acsfont{#1}\hfill}
\fi%
%
% ****************************************************************************************************


% ****************************************************************************************************
% 4. Setup floats: tables, (sub)figures, and captions
% ****************************************************************************************************
\usepackage{tabularx} % better tables
	\setlength{\extrarowheight}{3pt} % increase table row height
\newcommand{\tableheadline}[1]{\multicolumn{1}{c}{\spacedlowsmallcaps{#1}}}
\newcommand{\myfloatalign}{\centering} % to be used with each float for alignment
\usepackage{caption}
\DeclareCaptionLabelFormat{spacedlowsmallcaps}{%
  \bothIfFirst{\spacedlowsmallcaps{#1}}{~}\spacedlowsmallcaps{#2}}
\DeclareCaptionLabelSeparator*{periodenspace}{.\enspace}
\captionsetup{%
    hypcap=true,%
    format=plain,%
    indention=0cm,%
    font={small},% ,stretch=1.05
    labelformat=spacedlowsmallcaps,%
    labelsep=periodenspace}
\usepackage{subfig}
\usepackage{rotating}

\usepackage{chngcntr}
% \counterwithin{figure}{section}

% Add padding below caption
% \setlength{\belowcaptionskip}{3pt}

% ****************************************************************************************************
% 4b. Other useful packages
% ****************************************************************************************************
% If you don't have any of these, you can find them on CTAN
%
%\usepackage{bibspacing}     % Don't need to use this, I guess
%
\usepackage{pdflscape}       % Lets you put pages into landscape
\usepackage{rotating}        % Lets you rotate tables and figures
%
\usepackage{keyval}          % Key-value decoder (Part of the graphics bundle, so probably already included)
%
\usepackage{glossaries}      % Comprehensive glossary package
%
% ********************************************************************
% Things for tables
% ********************************************************************
\usepackage{array}           % Extended version of array and table environments
\usepackage{longtable}       % For tables spanning multiple pages
\usepackage{multirow}        % Items can span multiple table rows/cols
\usepackage{dcolumn}                % Lets you align decimal points within columns of a table
\newcolumntype{d}[1]{D{.}{.}{#1}}   % Adds decimal point column type. Must specify the number of decimal places as arg.
% NB: no need to import the (essential) booktabs package, because
% classicthesis.sty does this for us
%
% ********************************************************************
% Laying out text nicely
% ********************************************************************
% Things for maths
\usepackage{amssymb,amstext} % Full math equation support
\usepackage{amsthm}          % Better theorem environments
\usepackage{amsfonts}        % For blackboard bold, etc
\usepackage{nicefrac}        % Nicer fractions
%
%\usepackage{units}           % nice units, for 10Hz with a thin space, etc
\usepackage{siunitx}         % like units, but better
\DeclareSIUnit\cpd{cpd}      % cycles per degree as a cpd unit
\DeclareSIUnit\dva{dva}      % degrees of visual angle unit
\sisetup{separate-uncertainty = true, retain-explicit-plus}
%
\usepackage{mhchem}          % For chemical formulae
%
\usepackage{soul}            % Provides hy­phen­at­able spacing
%
% For displaying text in special ways
\usepackage{verbatim}        % Useful for program listings
% NB: listings (in its own section below) is useful for program listings too.
\usepackage{framed}          % Framed and/or shaded regions
\usepackage{color}           % Set text color
\definecolor{shadecolor}{gray}{0.9} % and make a new named color
%
% ********************************************************************
% Things for including images
% ********************************************************************
\usepackage{grffile}         % allow dots in middle of filenames
\usepackage{epstopdf}        % automatically convert eps files to pdf files
% when using epstopdf, on command line you must call pdflatex with arguments:
%--shell-escape --enable-write18
% NB: we also use the svg package, but that must be done later, after including
% the classicthesis class.
%
% ********************************************************************
% Moving things around
% ********************************************************************
\usepackage{calc}            % Allows raisebox, can move baselines
\usepackage{placeins}        % Provides \FloatBarrier which is an impass for figures 
\usepackage{float}           % Improved float interface
\usepackage{setspace}        % Can set space between lines
%
% ****************************************************************************************************

% ****************************************************************************************************
% 4c. Custom commands
% ****************************************************************************************************
\newcommand{\lyxdot}{.}
\newcommand{\mm}[0]{$\mathrm{\mu m}$ }
%\newcommand{\degree}{\ensuremath{^\circ}}
\newcommand{\ih}[0]{$I_{h}$ }
\newcommand{\cm}[0]{$cm^{2}$}
\newcommand{\sq}[0]{$^{2}$ }
% ********************************************************************
% Maths things
% ********************************************************************
% Declare a font which is useful for making symbols as curly letters,
% such as \mathpzc{H}
\DeclareMathAlphabet{\mathpzc}{OT1}{pzc}{m}{it}
%%% `Log-like' maths functions
\DeclareMathOperator{\sgn}{sgn}                         % sign
\DeclareMathOperator{\E}{\mathop{\mathbb E\/}}         % expectation
\DeclareMathOperator*{\EE}{\mathlarger{\mathlarger{\mathop{\mathbb E\/}}}}   % expectation
%\newcommand{\EE}{\mathlarger{\operatorname{\mathbb E}}}
\DeclareMathOperator{\PP}{\mathbb P\/}  % probability
%
% Derivatives
\let\underdot=\d                                        % rename builtin command \d{} to \underdot{}
%\renewcommand{\d}{\operatorname{d}}                     % old method
\renewcommand{\d}{\ensuremath{\operatorname{d}\!}}      % straight operator d (\! for no space after)
%\renewcommand{\d}{\ensuremath{d}}                       % italic (variable-like) d
\newcommand{\od}[1]{\frac{\d}{\d#1}}                    % first order ordinary derivative operator
\newcommand{\odn}[2]{\frac{\d^{#2}}{\d#1^{#2}}}         % n-th order ordinary derivative operator
\newcommand{\pd}[1]{\frac{\partial}{\partial #1}}       % first order partial derivative operator
\newcommand{\pdn}[2]{\frac{\partial^{#2}}{\partial #1^{#2}}}    % n-th order partial derivative operator
%
%%% Ordinals
%\newcommand{\ord}[2]{#1#2}                              %where we are doing normal 1st, 2nd, 3rd, 4th and not n-th
\newcommand{\ord}[2]{\ensuremath{\text{#1}^\text{#2}}}  %where we are doing normal 1st, 2nd, 3rd, 4th and not n-th
%\newcommand{\nth}[1]{#1\text{-th}}                      %where the argument is mathematical and we are in math-mode
%\newcommand{\mth}[1]{$#1$-th}                           %where the argument is mathematical but we are not in math-mode
\newcommand{\nth}[1]{\ensuremath{#1\text{-th}}}         %NEW: where argument is mathematical (may be used in any mode)
\newcommand{\mth}[1]{\nth{#1}}                          %ditto. for backward compatibility.
%
%%% Vectors
\newcommand{\mtx}[1]{\left[ \begin{matrix} #1 \end{matrix} \right]} %matrix
\newcommand{\col}[1]{\mtx{#1}}                                      %column vector
\newcommand{\row}[1]{[#1]}                                          %row vector
\newcommand{\tcol}[1]{\row{#1}^T}                                   %transposed col vector
%
%%% Notation
\newcommand{\VEC}[1]{\mathbf{#1}}                                   %vector symbol typeface
\newcommand{\SET}[1]{\mathbf{#1}}                                   %set symbol typeface
\newcommand{\NSYS}[1]{\mathbb{#1}}                                  %number system typeface (R, C, Z, N)
%
\newcommand{\R}{\NSYS{R}}                                           %number system typeface (R, C, Z, N)
\newcommand{\C}{\NSYS{C}}                                           %number system typeface (R, C, Z, N)
\newcommand{\Z}{\NSYS{Z}}                                           %number system typeface (R, C, Z, N)
\newcommand{\N}{\NSYS{N}}                                           %number system typeface (R, C, Z, N)
%
%% Imaginary unit
\newcommand{\iu}{{\mathrm{i}\mkern1mu}}
%% Real and Imaginary components
\renewcommand{\Re}{\operatorname{Re}}
\renewcommand{\Im}{\operatorname{Im}}
%
% ****************************************************************************************************

% ****************************************************************************************************
% 5. Setup code listings
% ****************************************************************************************************
\usepackage{listings} 
%\lstset{emph={trueIndex,root},emphstyle=\color{BlueViolet}}%\underbar} % for special keywords
\lstset{language=[LaTeX]Tex,%C++,
    keywordstyle=\color{RoyalBlue},%\bfseries,
    basicstyle=\small\ttfamily,
    %identifierstyle=\color{NavyBlue},
    commentstyle=\color{Green}\ttfamily,
    stringstyle=\rmfamily,
    numbers=none,%left,%
    numberstyle=\scriptsize,%\tiny
    stepnumber=5,
    numbersep=8pt,
    showstringspaces=false,
    breaklines=true,
    frameround=ftff,
    frame=single,
    belowcaptionskip=.75\baselineskip
    %frame=L
} 
% ****************************************************************************************************    		   


% ****************************************************************************************************
% 6. PDFLaTeX, hyperreferences and citation backreferences
% ****************************************************************************************************
% ********************************************************************
% Using PDFLaTeX
% ********************************************************************
\PassOptionsToPackage{pdftex,hyperfootnotes=false,pdfpagelabels,backref=page}{hyperref}
	\usepackage{hyperref}  % backref linktocpage pagebackref

\pdfcompresslevel=9
\pdfadjustspacing=1 
\PassOptionsToPackage{pdftex}{graphicx}
	\usepackage{graphicx} 

% ********************************************************************
% Setup the style of the backrefs from the bibliography
% (translate the options to any language you use)
% ********************************************************************
\newcommand{\backrefnotcitedstring}{\relax}%(Not cited.)
\newcommand{\backrefcitedsinglestring}[1]{(Cited on page~#1.)}
\newcommand{\backrefcitedmultistring}[1]{(Cited on pages~#1.)}
\ifthenelse{\boolean{enable-backrefs}}%
{%
		\PassOptionsToPackage{hyperpageref}{backref}
		\usepackage{backref} % to be loaded after hyperref package 
		   \renewcommand{\backreftwosep}{ and~} % separate 2 pages
		   \renewcommand{\backreflastsep}{, and~} % separate last of longer list
		   \renewcommand*{\backref}[1]{}  % disable standard
		   \renewcommand*{\backrefalt}[4]{% detailed backref
		      \ifcase #1 %
		         \backrefnotcitedstring%
		      \or%
		         \backrefcitedsinglestring{#2}%
		      \else%
		         \backrefcitedmultistring{#2}%
		      \fi}%
}{\relax}    

% ********************************************************************
% Hyperreferences
% ********************************************************************
\definecolor{webblue}{rgb}{0,0,0.930}
\hypersetup{%
    %draft,	% = no hyperlinking at all (useful in b/w printouts)
    colorlinks=true, linktocpage=true, pdfstartpage=3, pdfstartview=FitV,%
    % uncomment the following line if you want to have black links (e.g., for printing)
    %colorlinks=false, linktocpage=false, pdfborder={0 0 0}, pdfstartpage=3, pdfstartview=FitV,% 
    breaklinks=true, pdfpagemode=UseNone, pageanchor=true, pdfpagemode=UseOutlines,%
    plainpages=false, bookmarksnumbered, bookmarksopen=true, bookmarksopenlevel=1,%
    hypertexnames=true, pdfhighlight=/O,%nesting=true,%frenchlinks,%
    urlcolor=webbrown, linkcolor=RoyalBlue, citecolor=webgreen, %pagecolor=RoyalBlue,%
    %urlcolor=Black, linkcolor=Black, citecolor=Black, %pagecolor=Black,%
    pdftitle={\myTitle},%
    pdfauthor={\textcopyright\ \myName, \myUni, \myFaculty},%
    pdfsubject={},%
    pdfkeywords={},%
    pdfcreator={pdfLaTeX},%
    pdfproducer={LaTeX with hyperref and classicthesis}%
}   

% ********************************************************************
% Setup autoreferences
% ********************************************************************
% There are some issues regarding autorefnames
% http://www.ureader.de/msg/136221647.aspx
% http://www.tex.ac.uk/cgi-bin/texfaq2html?label=latexwords
% you have to redefine the makros for the 
% language you use, e.g., american, ngerman
% (as chosen when loading babel/AtBeginDocument)
% ********************************************************************
\makeatletter
\@ifpackageloaded{babel}%
    {%
       \addto\extrasamerican{%
					\renewcommand*{\figureautorefname}{Figure}%
					\renewcommand*{\tableautorefname}{Table}%
					\renewcommand*{\partautorefname}{Part}%
					\renewcommand*{\chapterautorefname}{Chapter}%
					\renewcommand*{\sectionautorefname}{Section}%
					\renewcommand*{\subsectionautorefname}{Section}%
					\renewcommand*{\subsubsectionautorefname}{Section}% 	
				}%
       \addto\extrasngerman{% 
					\renewcommand*{\paragraphautorefname}{Absatz}%
					\renewcommand*{\subparagraphautorefname}{Unterabsatz}%
					\renewcommand*{\footnoteautorefname}{Fu\"snote}%
					\renewcommand*{\FancyVerbLineautorefname}{Zeile}%
					\renewcommand*{\theoremautorefname}{Theorem}%
					\renewcommand*{\appendixautorefname}{Anhang}%
					\renewcommand*{\equationautorefname}{Gleichung}%        
					\renewcommand*{\itemautorefname}{Punkt}%
				}%	
			% Fix to getting autorefs for subfigures right (thanks to Belinda Vogt for changing the definition)
			\providecommand{\subfigureautorefname}{\figureautorefname}%  			
    }{\relax}
\makeatother


% ****************************************************************************************************
% 7. Last calls before the bar closes
% ****************************************************************************************************
% ********************************************************************
% Development Stuff
% ********************************************************************
\listfiles
%\PassOptionsToPackage{l2tabu,orthodox,abort}{nag}
%	\usepackage{nag}
%\PassOptionsToPackage{warning, all}{onlyamsmath}
%	\usepackage{onlyamsmath}

% ********************************************************************
% Last, but not least...
% ********************************************************************
\usepackage{classicthesis} 
% ****************************************************************************************************

% ****************************************************************************************************
% 8. Further adjustments (experimental)
% ****************************************************************************************************
% ********************************************************************
% Import svg package
% ********************************************************************
% Need to include the svg package after classicthesis because it does not
% play nicely with importing several required packages too early
\usepackage{svg}             % can include svg files with \includesvg

% ********************************************************************
% Changing the text area
% ********************************************************************
% \linespread{1.3}
\linespread{1.05} % a bit more for Palatino
%\areaset[current]{312pt}{761pt} % 686 (factor 2.2) + 33 head + 42 head \the\footskip
%\setlength{\marginparwidth}{7em}%
%\setlength{\marginparsep}{2em}%

% ********************************************************************
% Using different fonts
% ********************************************************************
%\usepackage[oldstylenums]{kpfonts} % oldstyle notextcomp
%\usepackage[osf]{libertine}
%\usepackage{hfoldsty} % Computer Modern with osf
%\usepackage[light,condensed,math]{iwona}
%\renewcommand{\sfdefault}{iwona}
%\usepackage{lmodern} % <-- no osf support :-(
%\usepackage[urw-garamond]{mathdesign} <-- no osf support :-(

% ********************************************************************
% Adjust colour of acronyms
% ********************************************************************
\makeatletter
\AtBeginDocument{%
  \renewcommand*{\AC@hyperlink}[2]{%
    \begingroup
      \hypersetup{linkcolor=Black}%webbrown
      \hyperlink{#1}{#2}%
    \endgroup
  }%
}
\makeatother
% ****************************************************************************************************

% ****************************************************************************************************  
% If you like the classicthesis, then I would appreciate a postcard. 
% My address can be found in the file ClassicThesis.pdf. A collection 
% of the postcards I received so far is available online at 
% http://postcards.miede.de
% ****************************************************************************************************

% ****************************************************************************************************
% 1. Configure classicthesis for your needs here, e.g., remove "drafting" below 
% in order to deactivate the time-stamp on the pages
% ****************************************************************************************************
\PassOptionsToPackage{eulerchapternumbers,listings,%drafting,%
                 pdfspacing,floatperchapter,%linedheaders,%
                 subfig,beramono,parts,%eulermath,
                 dottedtoc}{classicthesis}
% ********************************************************************
% Available options for classicthesis.sty 
% (see ClassicThesis.pdf for more information):
% drafting
% parts nochapters linedheaders
% eulerchapternumbers beramono eulermath pdfspacing minionprospacing
% tocaligned dottedtoc manychapters
% listings floatperchapter subfig
% ********************************************************************

% ********************************************************************
% Triggers for this config
% ******************************************************************** 
\usepackage{ifthen}
\newboolean{enable-backrefs} % enable backrefs in the bibliography
\setboolean{enable-backrefs}{true} % true false
% ****************************************************************************************************


% ****************************************************************************************************
% 2. Personal data and user ad-hoc commands
% ****************************************************************************************************
\newcommand{\myTitle}{Decoding information from neural populations in the visual cortex\xspace}
\newcommand{\mySubtitle}{\xspace}
\newcommand{\myDegree}{Doctor of Philosophy\xspace}
\newcommand{\myName}{Scott C. Lowe\xspace}
\newcommand{\myProf}{Prof. Mark van Rossum,\space\space{University of Edinburgh}\xspace} %Institute for Adaptive and Neural Computation,
\newcommand{\myOtherProf}{Prof. Stefano Panzeri,\space\space{Istituto Italiano di Technologia}\xspace} %Center for Neuroscience and Cognitive Systems,
\newcommand{\myThirdProf}{Prof. Alex Thiele,\space\space{Newcastle University}\xspace} %Institute of Neuroscience,
\newcommand{\mySupervisor}{SUPERVISORNAME\xspace}
\newcommand{\myFaculty}{Institute for Adaptive and Neural Computation\xspace}
\newcommand{\myDepartment}{School of Informatics\xspace}
\newcommand{\myUni}{University of Edinburgh\xspace}
\newcommand{\myLocation}{Edinburgh\xspace}
\newcommand{\myTime}{\the\year\xspace}
\newcommand{\myVersion}{version 4.1\xspace}

% ********************************************************************
% Setup, finetuning, and useful commands
% ********************************************************************
\newcounter{dummy} % necessary for correct hyperlinks (to index, bib, etc.)
\newlength{\abcd} % for ab..z string length calculation
\providecommand{\mLyX}{L\kern-.1667em\lower.25em\hbox{Y}\kern-.125emX\@}
\newcommand{\ie}{i.\,e.}
\newcommand{\Ie}{I.\,e.}
\newcommand{\eg}{e.\,g.}
\newcommand{\Eg}{E.\,g.} 
\newcommand{\etal}{\textit{et al.}}
\newcommand{\NB}{{N.B.}}
% ****************************************************************************************************


% ****************************************************************************************************
% 3. Loading some handy packages
% ****************************************************************************************************
% ******************************************************************** 
% Packages with options that might require adjustments
% ******************************************************************** 
\PassOptionsToPackage{latin9}{inputenc}	% latin9 (ISO-8859-9) = latin1+"Euro sign"
 \usepackage{inputenc}				

%\PassOptionsToPackage{ngerman,american}{babel}   % change this to your language(s)
% Spanish languages need extra options in order to work with this template
%\PassOptionsToPackage{spanish,es-lcroman}{babel}
 \usepackage{babel}					

%\PassOptionsToPackage{square,numbers}{natbib}
 \usepackage[natbibapa]{apacite}
 \usepackage{natbib,natbibspacing}

\PassOptionsToPackage{fleqn}{amsmath}		% math environments and more by the AMS 
 \usepackage{amsmath}

\usepackage{textgreek}

% ******************************************************************** 
% General useful packages
% ******************************************************************** 
\PassOptionsToPackage{T1}{fontenc} % T2A for cyrillics
	\usepackage{fontenc}     
\usepackage{textcomp} % fix warning with missing font shapes
\usepackage{scrhack} % fix warnings when using KOMA with listings package          
\usepackage{xspace} % to get the spacing after macros right  
\usepackage{mparhack} % get marginpar right
\usepackage{fixltx2e} % fixes some LaTeX stuff 
\PassOptionsToPackage{smaller}{acronym} % printonlyused,
	\usepackage{acronym} % nice macros for handling all acronyms in the thesis
%\renewcommand*{\acsfont}[1]{\textssc{#1}} % for MinionPro
%
\ifcsname/bflabel\endcsname%
    % For older versions of acronym
	\renewcommand{\bflabel}[1]{{#1}\hfill} % fix the list of acronyms
\else%
	% For acronym version >=1.41
	% Prevent acronym from using bold face
	\renewcommand{\aclabelfont}[1]{\acsfont{#1}\hfill}
\fi%
%
% ****************************************************************************************************


% ****************************************************************************************************
% 4. Setup floats: tables, (sub)figures, and captions
% ****************************************************************************************************
\usepackage{tabularx} % better tables
	\setlength{\extrarowheight}{3pt} % increase table row height
\newcommand{\tableheadline}[1]{\multicolumn{1}{c}{\spacedlowsmallcaps{#1}}}
\newcommand{\myfloatalign}{\centering} % to be used with each float for alignment
\usepackage{caption}
\DeclareCaptionLabelFormat{spacedlowsmallcaps}{%
  \bothIfFirst{\spacedlowsmallcaps{#1}}{~}\spacedlowsmallcaps{#2}}
\DeclareCaptionLabelSeparator*{periodenspace}{.\enspace}
\captionsetup{%
    hypcap=true,%
    format=plain,%
    indention=0cm,%
    font={small},% ,stretch=1.05
    labelformat=spacedlowsmallcaps,%
    labelsep=periodenspace}
\usepackage{subfig}
\usepackage{rotating}

\usepackage{chngcntr}
% \counterwithin{figure}{section}

% Add padding below caption
% \setlength{\belowcaptionskip}{3pt}

% ****************************************************************************************************
% 4b. Other useful packages
% ****************************************************************************************************
% If you don't have any of these, you can find them on CTAN
%
%\usepackage{bibspacing}     % Don't need to use this, I guess
%
\usepackage{pdflscape}       % Lets you put pages into landscape
\usepackage{rotating}        % Lets you rotate tables and figures
%
\usepackage{keyval}          % Key-value decoder (Part of the graphics bundle, so probably already included)
%
\usepackage{glossaries}      % Comprehensive glossary package
%
% ********************************************************************
% Things for tables
% ********************************************************************
\usepackage{array}           % Extended version of array and table environments
\usepackage{longtable}       % For tables spanning multiple pages
\usepackage{multirow}        % Items can span multiple table rows/cols
\usepackage{dcolumn}                % Lets you align decimal points within columns of a table
\newcolumntype{d}[1]{D{.}{.}{#1}}   % Adds decimal point column type. Must specify the number of decimal places as arg.
% NB: no need to import the (essential) booktabs package, because
% classicthesis.sty does this for us
%
% ********************************************************************
% Laying out text nicely
% ********************************************************************
% Things for maths
\usepackage{amssymb,amstext} % Full math equation support
\usepackage{amsthm}          % Better theorem environments
\usepackage{amsfonts}        % For blackboard bold, etc
\usepackage{nicefrac}        % Nicer fractions
%
%\usepackage{units}           % nice units, for 10Hz with a thin space, etc
\usepackage{siunitx}         % like units, but better
\DeclareSIUnit\cpd{cpd}      % cycles per degree as a cpd unit
\DeclareSIUnit\dva{dva}      % degrees of visual angle unit
\sisetup{separate-uncertainty = true, retain-explicit-plus}
%
\usepackage{mhchem}          % For chemical formulae
%
\usepackage{soul}            % Provides hy­phen­at­able spacing
%
% For displaying text in special ways
\usepackage{verbatim}        % Useful for program listings
% NB: listings (in its own section below) is useful for program listings too.
\usepackage{framed}          % Framed and/or shaded regions
\usepackage{color}           % Set text color
\definecolor{shadecolor}{gray}{0.9} % and make a new named color
%
% ********************************************************************
% Things for including images
% ********************************************************************
\usepackage{grffile}         % allow dots in middle of filenames
\usepackage{epstopdf}        % automatically convert eps files to pdf files
% when using epstopdf, on command line you must call pdflatex with arguments:
%--shell-escape --enable-write18
% NB: we also use the svg package, but that must be done later, after including
% the classicthesis class.
%
% ********************************************************************
% Moving things around
% ********************************************************************
\usepackage{calc}            % Allows raisebox, can move baselines
\usepackage{placeins}        % Provides \FloatBarrier which is an impass for figures 
\usepackage{float}           % Improved float interface
\usepackage{setspace}        % Can set space between lines
%
% ****************************************************************************************************

% ****************************************************************************************************
% 4c. Custom commands
% ****************************************************************************************************
\newcommand{\lyxdot}{.}
\newcommand{\mm}[0]{$\mathrm{\mu m}$ }
%\newcommand{\degree}{\ensuremath{^\circ}}
\newcommand{\ih}[0]{$I_{h}$ }
\newcommand{\cm}[0]{$cm^{2}$}
\newcommand{\sq}[0]{$^{2}$ }
% ********************************************************************
% Maths things
% ********************************************************************
% Declare a font which is useful for making symbols as curly letters,
% such as \mathpzc{H}
\DeclareMathAlphabet{\mathpzc}{OT1}{pzc}{m}{it}
%%% `Log-like' maths functions
\DeclareMathOperator{\sgn}{sgn}                         % sign
\DeclareMathOperator{\E}{\mathop{\mathbb E\/}}         % expectation
\DeclareMathOperator*{\EE}{\mathlarger{\mathlarger{\mathop{\mathbb E\/}}}}   % expectation
%\newcommand{\EE}{\mathlarger{\operatorname{\mathbb E}}}
\DeclareMathOperator{\PP}{\mathbb P\/}  % probability
%
% Derivatives
\let\underdot=\d                                        % rename builtin command \d{} to \underdot{}
%\renewcommand{\d}{\operatorname{d}}                     % old method
\renewcommand{\d}{\ensuremath{\operatorname{d}\!}}      % straight operator d (\! for no space after)
%\renewcommand{\d}{\ensuremath{d}}                       % italic (variable-like) d
\newcommand{\od}[1]{\frac{\d}{\d#1}}                    % first order ordinary derivative operator
\newcommand{\odn}[2]{\frac{\d^{#2}}{\d#1^{#2}}}         % n-th order ordinary derivative operator
\newcommand{\pd}[1]{\frac{\partial}{\partial #1}}       % first order partial derivative operator
\newcommand{\pdn}[2]{\frac{\partial^{#2}}{\partial #1^{#2}}}    % n-th order partial derivative operator
%
%%% Ordinals
%\newcommand{\ord}[2]{#1#2}                              %where we are doing normal 1st, 2nd, 3rd, 4th and not n-th
\newcommand{\ord}[2]{\ensuremath{\text{#1}^\text{#2}}}  %where we are doing normal 1st, 2nd, 3rd, 4th and not n-th
%\newcommand{\nth}[1]{#1\text{-th}}                      %where the argument is mathematical and we are in math-mode
%\newcommand{\mth}[1]{$#1$-th}                           %where the argument is mathematical but we are not in math-mode
\newcommand{\nth}[1]{\ensuremath{#1\text{-th}}}         %NEW: where argument is mathematical (may be used in any mode)
\newcommand{\mth}[1]{\nth{#1}}                          %ditto. for backward compatibility.
%
%%% Vectors
\newcommand{\mtx}[1]{\left[ \begin{matrix} #1 \end{matrix} \right]} %matrix
\newcommand{\col}[1]{\mtx{#1}}                                      %column vector
\newcommand{\row}[1]{[#1]}                                          %row vector
\newcommand{\tcol}[1]{\row{#1}^T}                                   %transposed col vector
%
%%% Notation
\newcommand{\VEC}[1]{\mathbf{#1}}                                   %vector symbol typeface
\newcommand{\SET}[1]{\mathbf{#1}}                                   %set symbol typeface
\newcommand{\NSYS}[1]{\mathbb{#1}}                                  %number system typeface (R, C, Z, N)
%
\newcommand{\R}{\NSYS{R}}                                           %number system typeface (R, C, Z, N)
\newcommand{\C}{\NSYS{C}}                                           %number system typeface (R, C, Z, N)
\newcommand{\Z}{\NSYS{Z}}                                           %number system typeface (R, C, Z, N)
\newcommand{\N}{\NSYS{N}}                                           %number system typeface (R, C, Z, N)
%
%% Imaginary unit
\newcommand{\iu}{{\mathrm{i}\mkern1mu}}
%% Real and Imaginary components
\renewcommand{\Re}{\operatorname{Re}}
\renewcommand{\Im}{\operatorname{Im}}
%
% ****************************************************************************************************

% ****************************************************************************************************
% 5. Setup code listings
% ****************************************************************************************************
\usepackage{listings} 
%\lstset{emph={trueIndex,root},emphstyle=\color{BlueViolet}}%\underbar} % for special keywords
\lstset{language=[LaTeX]Tex,%C++,
    keywordstyle=\color{RoyalBlue},%\bfseries,
    basicstyle=\small\ttfamily,
    %identifierstyle=\color{NavyBlue},
    commentstyle=\color{Green}\ttfamily,
    stringstyle=\rmfamily,
    numbers=none,%left,%
    numberstyle=\scriptsize,%\tiny
    stepnumber=5,
    numbersep=8pt,
    showstringspaces=false,
    breaklines=true,
    frameround=ftff,
    frame=single,
    belowcaptionskip=.75\baselineskip
    %frame=L
} 
% ****************************************************************************************************    		   


% ****************************************************************************************************
% 6. PDFLaTeX, hyperreferences and citation backreferences
% ****************************************************************************************************
% ********************************************************************
% Using PDFLaTeX
% ********************************************************************
\PassOptionsToPackage{pdftex,hyperfootnotes=false,pdfpagelabels,backref=page}{hyperref}
	\usepackage{hyperref}  % backref linktocpage pagebackref

\pdfcompresslevel=9
\pdfadjustspacing=1 
\PassOptionsToPackage{pdftex}{graphicx}
	\usepackage{graphicx} 

% ********************************************************************
% Setup the style of the backrefs from the bibliography
% (translate the options to any language you use)
% ********************************************************************
\newcommand{\backrefnotcitedstring}{\relax}%(Not cited.)
\newcommand{\backrefcitedsinglestring}[1]{(Cited on page~#1.)}
\newcommand{\backrefcitedmultistring}[1]{(Cited on pages~#1.)}
\ifthenelse{\boolean{enable-backrefs}}%
{%
		\PassOptionsToPackage{hyperpageref}{backref}
		\usepackage{backref} % to be loaded after hyperref package 
		   \renewcommand{\backreftwosep}{ and~} % separate 2 pages
		   \renewcommand{\backreflastsep}{, and~} % separate last of longer list
		   \renewcommand*{\backref}[1]{}  % disable standard
		   \renewcommand*{\backrefalt}[4]{% detailed backref
		      \ifcase #1 %
		         \backrefnotcitedstring%
		      \or%
		         \backrefcitedsinglestring{#2}%
		      \else%
		         \backrefcitedmultistring{#2}%
		      \fi}%
}{\relax}    

% ********************************************************************
% Hyperreferences
% ********************************************************************
\definecolor{webblue}{rgb}{0,0,0.930}
\hypersetup{%
    %draft,	% = no hyperlinking at all (useful in b/w printouts)
    colorlinks=true, linktocpage=true, pdfstartpage=3, pdfstartview=FitV,%
    % uncomment the following line if you want to have black links (e.g., for printing)
    %colorlinks=false, linktocpage=false, pdfborder={0 0 0}, pdfstartpage=3, pdfstartview=FitV,% 
    breaklinks=true, pdfpagemode=UseNone, pageanchor=true, pdfpagemode=UseOutlines,%
    plainpages=false, bookmarksnumbered, bookmarksopen=true, bookmarksopenlevel=1,%
    hypertexnames=true, pdfhighlight=/O,%nesting=true,%frenchlinks,%
    urlcolor=webbrown, linkcolor=RoyalBlue, citecolor=webgreen, %pagecolor=RoyalBlue,%
    %urlcolor=Black, linkcolor=Black, citecolor=Black, %pagecolor=Black,%
    pdftitle={\myTitle},%
    pdfauthor={\textcopyright\ \myName, \myUni, \myFaculty},%
    pdfsubject={},%
    pdfkeywords={},%
    pdfcreator={pdfLaTeX},%
    pdfproducer={LaTeX with hyperref and classicthesis}%
}   

% ********************************************************************
% Setup autoreferences
% ********************************************************************
% There are some issues regarding autorefnames
% http://www.ureader.de/msg/136221647.aspx
% http://www.tex.ac.uk/cgi-bin/texfaq2html?label=latexwords
% you have to redefine the makros for the 
% language you use, e.g., american, ngerman
% (as chosen when loading babel/AtBeginDocument)
% ********************************************************************
\makeatletter
\@ifpackageloaded{babel}%
    {%
       \addto\extrasamerican{%
					\renewcommand*{\figureautorefname}{Figure}%
					\renewcommand*{\tableautorefname}{Table}%
					\renewcommand*{\partautorefname}{Part}%
					\renewcommand*{\chapterautorefname}{Chapter}%
					\renewcommand*{\sectionautorefname}{Section}%
					\renewcommand*{\subsectionautorefname}{Section}%
					\renewcommand*{\subsubsectionautorefname}{Section}% 	
				}%
       \addto\extrasngerman{% 
					\renewcommand*{\paragraphautorefname}{Absatz}%
					\renewcommand*{\subparagraphautorefname}{Unterabsatz}%
					\renewcommand*{\footnoteautorefname}{Fu\"snote}%
					\renewcommand*{\FancyVerbLineautorefname}{Zeile}%
					\renewcommand*{\theoremautorefname}{Theorem}%
					\renewcommand*{\appendixautorefname}{Anhang}%
					\renewcommand*{\equationautorefname}{Gleichung}%        
					\renewcommand*{\itemautorefname}{Punkt}%
				}%	
			% Fix to getting autorefs for subfigures right (thanks to Belinda Vogt for changing the definition)
			\providecommand{\subfigureautorefname}{\figureautorefname}%  			
    }{\relax}
\makeatother


% ****************************************************************************************************
% 7. Last calls before the bar closes
% ****************************************************************************************************
% ********************************************************************
% Development Stuff
% ********************************************************************
\listfiles
%\PassOptionsToPackage{l2tabu,orthodox,abort}{nag}
%	\usepackage{nag}
%\PassOptionsToPackage{warning, all}{onlyamsmath}
%	\usepackage{onlyamsmath}

% ********************************************************************
% Last, but not least...
% ********************************************************************
\usepackage{classicthesis} 
% ****************************************************************************************************

% ****************************************************************************************************
% 8. Further adjustments (experimental)
% ****************************************************************************************************
% ********************************************************************
% Import svg package
% ********************************************************************
% Need to include the svg package after classicthesis because it does not
% play nicely with importing several required packages too early
\usepackage{svg}             % can include svg files with \includesvg

% ********************************************************************
% Changing the text area
% ********************************************************************
% \linespread{1.3}
\linespread{1.05} % a bit more for Palatino
%\areaset[current]{312pt}{761pt} % 686 (factor 2.2) + 33 head + 42 head \the\footskip
%\setlength{\marginparwidth}{7em}%
%\setlength{\marginparsep}{2em}%

% ********************************************************************
% Using different fonts
% ********************************************************************
%\usepackage[oldstylenums]{kpfonts} % oldstyle notextcomp
%\usepackage[osf]{libertine}
%\usepackage{hfoldsty} % Computer Modern with osf
%\usepackage[light,condensed,math]{iwona}
%\renewcommand{\sfdefault}{iwona}
%\usepackage{lmodern} % <-- no osf support :-(
%\usepackage[urw-garamond]{mathdesign} <-- no osf support :-(

% ********************************************************************
% Adjust colour of acronyms
% ********************************************************************
\makeatletter
\AtBeginDocument{%
  \renewcommand*{\AC@hyperlink}[2]{%
    \begingroup
      \hypersetup{linkcolor=Black}%webbrown
      \hyperlink{#1}{#2}%
    \endgroup
  }%
}
\makeatother
% ****************************************************************************************************

% ****************************************************************************************************  
% If you like the classicthesis, then I would appreciate a postcard. 
% My address can be found in the file ClassicThesis.pdf. A collection 
% of the postcards I received so far is available online at 
% http://postcards.miede.de
% ****************************************************************************************************

% ****************************************************************************************************
% 1. Configure classicthesis for your needs here, e.g., remove "drafting" below 
% in order to deactivate the time-stamp on the pages
% ****************************************************************************************************
\PassOptionsToPackage{eulerchapternumbers,listings,%drafting,%
                 pdfspacing,floatperchapter,%linedheaders,%
                 subfig,beramono,parts,%eulermath,
                 dottedtoc}{classicthesis}
% ********************************************************************
% Available options for classicthesis.sty 
% (see ClassicThesis.pdf for more information):
% drafting
% parts nochapters linedheaders
% eulerchapternumbers beramono eulermath pdfspacing minionprospacing
% tocaligned dottedtoc manychapters
% listings floatperchapter subfig
% ********************************************************************

% ********************************************************************
% Triggers for this config
% ******************************************************************** 
\usepackage{ifthen}
\newboolean{enable-backrefs} % enable backrefs in the bibliography
\setboolean{enable-backrefs}{false} % true false
% ****************************************************************************************************


% ****************************************************************************************************
% 2. Personal data and user ad-hoc commands
% ****************************************************************************************************
\newcommand{\myTitle}{Decoding information in the visual cortex\xspace}
\newcommand{\mySubtitle}{\xspace}
\newcommand{\myDegree}{Doctor of Philosophy\xspace}
\newcommand{\myName}{Scott C. Lowe\xspace}
\newcommand{\myProf}{Mark van Rossum\xspace}
\newcommand{\myOtherProf}{Stefano Panzeri\xspace}
\newcommand{\myThirdProf}{Alex Thiele\xspace}
\newcommand{\mySupervisor}{SUPERVISORNAME\xspace}
\newcommand{\myFaculty}{Institute for Adaptive and Neural Computation\xspace}
\newcommand{\myDepartment}{School of Informatics\xspace}
\newcommand{\myUni}{University of Edinburgh\xspace}
\newcommand{\myLocation}{Edinburgh\xspace}
\newcommand{\myTime}{\the\year\xspace}
\newcommand{\myVersion}{version 4.1\xspace}

% ********************************************************************
% Setup, finetuning, and useful commands
% ********************************************************************
\newcounter{dummy} % necessary for correct hyperlinks (to index, bib, etc.)
\newlength{\abcd} % for ab..z string length calculation
\providecommand{\mLyX}{L\kern-.1667em\lower.25em\hbox{Y}\kern-.125emX\@}
\newcommand{\ie}{i.\,e.}
\newcommand{\Ie}{I.\,e.}
\newcommand{\eg}{e.\,g.}
\newcommand{\Eg}{E.\,g.} 
\newcommand{\etal}{\textit{et al.}}
\newcommand{\NB}{{N.B.}}
% ****************************************************************************************************


% ****************************************************************************************************
% 3. Loading some handy packages
% ****************************************************************************************************
% ******************************************************************** 
% Packages with options that might require adjustments
% ******************************************************************** 
\PassOptionsToPackage{latin9}{inputenc}	% latin9 (ISO-8859-9) = latin1+"Euro sign"
 \usepackage{inputenc}				

%\PassOptionsToPackage{ngerman,american}{babel}   % change this to your language(s)
% Spanish languages need extra options in order to work with this template
%\PassOptionsToPackage{spanish,es-lcroman}{babel}
 \usepackage{babel}					

%\PassOptionsToPackage{square,numbers}{natbib}
 \usepackage{natbib,natbibspacing}

\PassOptionsToPackage{fleqn}{amsmath}		% math environments and more by the AMS 
 \usepackage{amsmath}

% ******************************************************************** 
% General useful packages
% ******************************************************************** 
\PassOptionsToPackage{T1}{fontenc} % T2A for cyrillics
	\usepackage{fontenc}     
\usepackage{textcomp} % fix warning with missing font shapes
\usepackage{scrhack} % fix warnings when using KOMA with listings package          
\usepackage{xspace} % to get the spacing after macros right  
\usepackage{mparhack} % get marginpar right
\usepackage{fixltx2e} % fixes some LaTeX stuff 
\PassOptionsToPackage{printonlyused,smaller}{acronym}
	\usepackage{acronym} % nice macros for handling all acronyms in the thesis
%\renewcommand*{\acsfont}[1]{\textssc{#1}} % for MinionPro
%
\ifcsname/bflabel\endcsname%
    % For older versions of acronym
	\renewcommand{\bflabel}[1]{{#1}\hfill} % fix the list of acronyms
\else%
	% For acronym version >=1.41
	% Prevent acronym from using bold face
	\renewcommand{\aclabelfont}[1]{\acsfont{#1}\hfill}
\fi%
%
% ****************************************************************************************************


% ****************************************************************************************************
% 4. Setup floats: tables, (sub)figures, and captions
% ****************************************************************************************************
\usepackage{tabularx} % better tables
	\setlength{\extrarowheight}{3pt} % increase table row height
\newcommand{\tableheadline}[1]{\multicolumn{1}{c}{\spacedlowsmallcaps{#1}}}
\newcommand{\myfloatalign}{\centering} % to be used with each float for alignment
\usepackage{caption}
\captionsetup{font=small,font=singlespacing,format=plain,indention=0cm} %font=sf,
\usepackage{subfig}  


% ****************************************************************************************************
% 4b. Other useful packages
% ****************************************************************************************************
% If you don't have any of these, you can find them on CTAN
%
%\usepackage{bibspacing}     % Don't need to use this, I guess
%
\usepackage{pdflscape}       % Lets you put pages into landscape
\usepackage{rotating}        % Lets you rotate tables and figures
%
\usepackage{keyval}          % Key-value decoder (Part of the graphics bundle, so probably already included)
%
\usepackage{glossaries}      % Comprehensive glossary package
%
% ********************************************************************
% Things for tables
% ********************************************************************
\usepackage{array}           % Extended version of array and table environments
\usepackage{longtable}       % For tables spanning multiple pages
\usepackage{multirow}        % Items can span multiple table rows/cols
\usepackage{dcolumn}                % Lets you align decimal points within columns of a table
\newcolumntype{d}[1]{D{.}{.}{#1}}   % Adds decimal point column type. Must specify the number of decimal places as arg.
% NB: no need to import the (essential) booktabs package, because
% classicthesis.sty does this for us
%
% ********************************************************************
% Laying out text nicely
% ********************************************************************
% Things for maths
\usepackage{amssymb,amstext} % Full math equation support
\usepackage{amsthm}          % Better theorem environments
\usepackage{amsfonts}        % For blackboard bold, etc
\usepackage{nicefrac}        % Nicer fractions
%
%\usepackage{units}           % nice units, for 10Hz with a thin space, etc
\usepackage{siunitx}         % like units, but better
\DeclareSIUnit\cpd{cpd}      % cycles per degree as a cpd unit
%
\usepackage{soul}            % Provides hy­phen­at­able spacing
%
% For displaying text in special ways
\usepackage{verbatim}        % Useful for program listings
% NB: listings (in its own section below) is useful for program listings too.
\usepackage{framed}          % Framed and/or shaded regions
\usepackage{color}           % Set text color
\definecolor{shadecolor}{gray}{0.9} % and make a new named color
%
% ********************************************************************
% Things for including images
% ********************************************************************
\usepackage{grffile}         % allow dots in middle of filenames
\usepackage{epstopdf}        % automatically convert eps files to pdf files
% when using epstopdf, on command line you must call pdflatex with arguments:
%--shell-escape --enable-write18
% NB: we also use the svg package, but that must be done later, after including
% the classicthesis class.
%
% ********************************************************************
% Moving things around
% ********************************************************************
\usepackage{calc}            % Allows raisebox, can move baselines
\usepackage{placeins}        % Provides \FloatBarrier which is an impass for figures 
\usepackage{float}           % Improved float interface
\usepackage{setspace}        % Can set space between lines
%
% ****************************************************************************************************

% ****************************************************************************************************
% 4c. Custom commands
% ****************************************************************************************************
\newcommand{\lyxdot}{.}
\newcommand{\mm}[0]{$\mathrm{\mu m}$ }
%\newcommand{\degree}{\ensuremath{^\circ}}
\newcommand{\ih}[0]{$I_{h}$ }
\newcommand{\cm}[0]{$cm^{2}$}
\newcommand{\sq}[0]{$^{2}$ }
% ********************************************************************
% Maths things
% ********************************************************************
% Declare a font which is useful for making symbols as curly letters,
% such as \mathpzc{H}
\DeclareMathAlphabet{\mathpzc}{OT1}{pzc}{m}{it}
%%% `Log-like' maths functions
\DeclareMathOperator{\sgn}{sgn}                         % sign
\DeclareMathOperator*{\E}{\mathop{\mathbb E\/}}         % expectation
%
% Derivatives
\let\underdot=\d                                        % rename builtin command \d{} to \underdot{}
%\renewcommand{\d}{\operatorname{d}}                     % old method
\renewcommand{\d}{\ensuremath{\operatorname{d}\!}}      % straight operator d (\! for no space after)
%\renewcommand{\d}{\ensuremath{d}}                       % italic (variable-like) d
\newcommand{\od}[1]{\frac{\d}{\d#1}}                    % first order ordinary derivative operator
\newcommand{\odn}[2]{\frac{\d^{#2}}{\d#1^{#2}}}         % n-th order ordinary derivative operator
\newcommand{\pd}[1]{\frac{\partial}{\partial #1}}       % first order partial derivative operator
\newcommand{\pdn}[2]{\frac{\partial^{#2}}{\partial #1^{#2}}}    % n-th order partial derivative operator
%
%%% Ordinals
%\newcommand{\ord}[2]{#1#2}                              %where we are doing normal 1st, 2nd, 3rd, 4th and not n-th
\newcommand{\ord}[2]{\ensuremath{\text{#1}^\text{#2}}}  %where we are doing normal 1st, 2nd, 3rd, 4th and not n-th
%\newcommand{\nth}[1]{#1\text{-th}}                      %where the argument is mathematical and we are in math-mode
%\newcommand{\mth}[1]{$#1$-th}                           %where the argument is mathematical but we are not in math-mode
\newcommand{\nth}[1]{\ensuremath{#1\text{-th}}}         %NEW: where argument is mathematical (may be used in any mode)
\newcommand{\mth}[1]{\nth{#1}}                          %ditto. for backward compatibility.
%
%%% Vectors
\newcommand{\mtx}[1]{\left[ \begin{matrix} #1 \end{matrix} \right]} %matrix
\newcommand{\col}[1]{\mtx{#1}}                                      %column vector
\newcommand{\row}[1]{[#1]}                                          %row vector
\newcommand{\tcol}[1]{\row{#1}^T}                                   %transposed col vector
%
%%% Notation
\newcommand{\VEC}[1]{\mathbf{#1}}                                   %vector symbol typeface
\newcommand{\SET}[1]{\mathbf{#1}}                                   %set symbol typeface
\newcommand{\NSYS}[1]{\mathbb{#1}}                                  %number system typeface (R, C, Z, N)
%
\newcommand{\R}{\NSYS{R}}                                           %number system typeface (R, C, Z, N)
\newcommand{\C}{\NSYS{C}}                                           %number system typeface (R, C, Z, N)
\newcommand{\Z}{\NSYS{Z}}                                           %number system typeface (R, C, Z, N)
\newcommand{\N}{\NSYS{N}}                                           %number system typeface (R, C, Z, N)
% ****************************************************************************************************

% ****************************************************************************************************
% 5. Setup code listings
% ****************************************************************************************************
\usepackage{listings} 
%\lstset{emph={trueIndex,root},emphstyle=\color{BlueViolet}}%\underbar} % for special keywords
\lstset{language=[LaTeX]Tex,%C++,
    keywordstyle=\color{RoyalBlue},%\bfseries,
    basicstyle=\small\ttfamily,
    %identifierstyle=\color{NavyBlue},
    commentstyle=\color{Green}\ttfamily,
    stringstyle=\rmfamily,
    numbers=none,%left,%
    numberstyle=\scriptsize,%\tiny
    stepnumber=5,
    numbersep=8pt,
    showstringspaces=false,
    breaklines=true,
    frameround=ftff,
    frame=single,
    belowcaptionskip=.75\baselineskip
    %frame=L
} 
% ****************************************************************************************************    		   


% ****************************************************************************************************
% 6. PDFLaTeX, hyperreferences and citation backreferences
% ****************************************************************************************************
% ********************************************************************
% Using PDFLaTeX
% ********************************************************************
\PassOptionsToPackage{pdftex,hyperfootnotes=false,pdfpagelabels}{hyperref}
	\usepackage{hyperref}  % backref linktocpage pagebackref
\pdfcompresslevel=9
\pdfadjustspacing=1 
\PassOptionsToPackage{pdftex}{graphicx}
	\usepackage{graphicx} 

% ********************************************************************
% Setup the style of the backrefs from the bibliography
% (translate the options to any language you use)
% ********************************************************************
\newcommand{\backrefnotcitedstring}{\relax}%(Not cited.)
\newcommand{\backrefcitedsinglestring}[1]{(Cited on page~#1.)}
\newcommand{\backrefcitedmultistring}[1]{(Cited on pages~#1.)}
\ifthenelse{\boolean{enable-backrefs}}%
{%
		\PassOptionsToPackage{hyperpageref}{backref}
		\usepackage{backref} % to be loaded after hyperref package 
		   \renewcommand{\backreftwosep}{ and~} % separate 2 pages
		   \renewcommand{\backreflastsep}{, and~} % separate last of longer list
		   \renewcommand*{\backref}[1]{}  % disable standard
		   \renewcommand*{\backrefalt}[4]{% detailed backref
		      \ifcase #1 %
		         \backrefnotcitedstring%
		      \or%
		         \backrefcitedsinglestring{#2}%
		      \else%
		         \backrefcitedmultistring{#2}%
		      \fi}%
}{\relax}    

% ********************************************************************
% Hyperreferences
% ********************************************************************
\hypersetup{%
    %draft,	% = no hyperlinking at all (useful in b/w printouts)
    colorlinks=true, linktocpage=true, pdfstartpage=3, pdfstartview=FitV,%
    % uncomment the following line if you want to have black links (e.g., for printing)
    %colorlinks=false, linktocpage=false, pdfborder={0 0 0}, pdfstartpage=3, pdfstartview=FitV,% 
    breaklinks=true, pdfpagemode=UseNone, pageanchor=true, pdfpagemode=UseOutlines,%
    plainpages=false, bookmarksnumbered, bookmarksopen=true, bookmarksopenlevel=1,%
    hypertexnames=true, pdfhighlight=/O,%nesting=true,%frenchlinks,%
    urlcolor=webbrown, linkcolor=RoyalBlue, citecolor=webgreen, %pagecolor=RoyalBlue,%
    %urlcolor=Black, linkcolor=Black, citecolor=Black, %pagecolor=Black,%
    pdftitle={\myTitle},%
    pdfauthor={\textcopyright\ \myName, \myUni, \myFaculty},%
    pdfsubject={},%
    pdfkeywords={},%
    pdfcreator={pdfLaTeX},%
    pdfproducer={LaTeX with hyperref and classicthesis}%
}   

% ********************************************************************
% Setup autoreferences
% ********************************************************************
% There are some issues regarding autorefnames
% http://www.ureader.de/msg/136221647.aspx
% http://www.tex.ac.uk/cgi-bin/texfaq2html?label=latexwords
% you have to redefine the makros for the 
% language you use, e.g., american, ngerman
% (as chosen when loading babel/AtBeginDocument)
% ********************************************************************
\makeatletter
\@ifpackageloaded{babel}%
    {%
       \addto\extrasamerican{%
					\renewcommand*{\figureautorefname}{Figure}%
					\renewcommand*{\tableautorefname}{Table}%
					\renewcommand*{\partautorefname}{Part}%
					\renewcommand*{\chapterautorefname}{Chapter}%
					\renewcommand*{\sectionautorefname}{Section}%
					\renewcommand*{\subsectionautorefname}{Section}%
					\renewcommand*{\subsubsectionautorefname}{Section}% 	
				}%
       \addto\extrasngerman{% 
					\renewcommand*{\paragraphautorefname}{Absatz}%
					\renewcommand*{\subparagraphautorefname}{Unterabsatz}%
					\renewcommand*{\footnoteautorefname}{Fu\"snote}%
					\renewcommand*{\FancyVerbLineautorefname}{Zeile}%
					\renewcommand*{\theoremautorefname}{Theorem}%
					\renewcommand*{\appendixautorefname}{Anhang}%
					\renewcommand*{\equationautorefname}{Gleichung}%        
					\renewcommand*{\itemautorefname}{Punkt}%
				}%	
			% Fix to getting autorefs for subfigures right (thanks to Belinda Vogt for changing the definition)
			\providecommand{\subfigureautorefname}{\figureautorefname}%  			
    }{\relax}
\makeatother


% ****************************************************************************************************
% 7. Last calls before the bar closes
% ****************************************************************************************************
% ********************************************************************
% Development Stuff
% ********************************************************************
\listfiles
%\PassOptionsToPackage{l2tabu,orthodox,abort}{nag}
%	\usepackage{nag}
%\PassOptionsToPackage{warning, all}{onlyamsmath}
%	\usepackage{onlyamsmath}

% ********************************************************************
% Last, but not least...
% ********************************************************************
\usepackage{classicthesis} 
% ****************************************************************************************************

% Need to include the svg package afterward classicthesis because it does not
% play nicely with importing several required packages too early
\usepackage{svg}             % can include svg files with \includesvg

% ****************************************************************************************************
% 8. Further adjustments (experimental)
% ****************************************************************************************************
% ********************************************************************
% Changing the text area
% ********************************************************************
\linespread{1.3}
%\linespread{1.05} % a bit more for Palatino
%\areaset[current]{312pt}{761pt} % 686 (factor 2.2) + 33 head + 42 head \the\footskip
%\setlength{\marginparwidth}{7em}%
%\setlength{\marginparsep}{2em}%

% ********************************************************************
% Using different fonts
% ********************************************************************
%\usepackage[oldstylenums]{kpfonts} % oldstyle notextcomp
%\usepackage[osf]{libertine}
%\usepackage{hfoldsty} % Computer Modern with osf
%\usepackage[light,condensed,math]{iwona}
%\renewcommand{\sfdefault}{iwona}
%\usepackage{lmodern} % <-- no osf support :-(
%\usepackage[urw-garamond]{mathdesign} <-- no osf support :-(
% ****************************************************************************************************
