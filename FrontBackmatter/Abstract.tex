%*******************************************************
% Abstract
%*******************************************************
%\renewcommand{\abstractname}{Abstract}

\begingroup

\let\clearpage\relax
\let\cleardoublepage\relax
\let\cleardoublepage\relax

\chapter*{Abstract}
\chaptermark{Abstract}
\addcontentsline{toc}{chapter}{\tocEntry{Abstract}}

Visual perception in mammals is made possible by the visual system and the visual cortex.
However, precisely how visual information is coded in the brain and how training can improve this encoding is unclear.

The ability to see and process visual information is not an innate property of the visual cortex.
Instead, it is learnt from exposure to visual stimuli.
We first considered how visual perception is learnt, by studying the perceptual learning of contrast discrimination in macaques.
We investigated how changes in population activity in the visual cortices \acs{V1} and \acs{V4} correlate with the changes in behavioural response during training on this task.
Our results indicate that changes in the learnt neural and behavioural responses are directed toward optimising the performance on the training task, rather than a general improvement in perception of the presented stimulus type.
We report that the most informative signal about the contrast of the stimulus within \acs{V1} and \acs{V4} is the transient stimulus-onset response in \acs{V1}, \SI{50}{\milli\second} after the stimulus presentation begins.
However, this signal does not become more informative with training, suggesting it is an innate and untrainable property of the system, on these timescales at least.
Using a linear decoder to classify the stimulus based on the population activity, we find that information in the \acs{V4} population is closely related to the information available to the higher cortical regions involved with decision making, since the performance of the decoder is similar to the performance of the animal throughout training.
These findings suggest that training the subject on this task directs \acs{V4} to improve its read out of contrast information contained in \acs{V1}, and cortical regions responsible for decision making use this to improve the performance with training.
The structure of noise correlations between the recorded neurons changes with training, but this does not appear to cause the increase in behavioural performance.
Furthermore, our results suggest there is feedback of information about the stimulus into the visual cortex after \SI{300}{\milli\second} of stimulus presentation, which may be related to the high-level percept of the stimulus within the brain.
After training on the task, but not before, information about the stimulus persists in the activity of both \acs{V1} and \acs{V4} at least \SI{400}{\milli\second} after the stimulus is removed.

In the second part, we explore how information is distributed across the anatomical layers of the visual cortex.
Cortical oscillations in the \acf{LFP} and \acf{CSD} within \acs{V1}, driven by population-level activity, are known to contain information about visual stimulation.
However the purpose of these oscillations, the sites where they originate, and what properties of the stimulus is encoded within them is still unknown.
By recording the \ac{LFP} at multiple recording sites along the cortical depth of macaque \acs{V1} during presentation of a natural movie stimulus, we investigated the structure of visual information encoded in cortical oscillations.
We found that despite a homogeneous distribution of the power of oscillations across the cortical depth, information was compartmentalised into the oscillations of the \SIrange{4}{16}{Hz} range at the \acl{G} (\acsu{G}, layer 4) depths and the \SIrange{60}{170}{Hz} range at the \acl{SG} (\acsu{SG}, layers 1--3) depths, the latter of which is redundant with the population-level firing rate.
These two frequency ranges contain independent information about the stimulus, which we identify as related to two spatiotemporal aspects of the visual stimulus.
Oscillations in the visual cortex with frequencies \SI{<40}{Hz} contain information about fast changes in low spatial frequency.
Frequencies \SI{>40}{Hz} and multi-unit firing rates contain information about properties of the stimulus related to changes, both slow and fast, at finer-grained spatial scales.
The spatiotemporal domains encoded in each are complementary.
In particular, both the power and phase of oscillations in the \SIrange{7}{20}{Hz} range contain information about scene transitions in the presented movie stimulus.
Such changes in the stimulus are similar to saccades in natural behaviour, and this may be indicative of predictive coding within the cortex.


\endgroup			

% \vfill
