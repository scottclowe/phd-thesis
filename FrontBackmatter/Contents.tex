%*******************************************************
% Table of Contents
%*******************************************************
%\phantomsection
\refstepcounter{dummy}
\pdfbookmark[1]{\contentsname}{tableofcontents}
\setcounter{tocdepth}{2} % <-- 2 includes up to subsections in the ToC
\setcounter{secnumdepth}{3} % <-- 3 numbers up to subsubsections
\manualmark
\markboth{\spacedlowsmallcaps{\contentsname}}{\spacedlowsmallcaps{\contentsname}}
\tableofcontents 
\automark[section]{chapter}
\renewcommand{\chaptermark}[1]{\markboth{\spacedlowsmallcaps{#1}}{\spacedlowsmallcaps{#1}}}
\renewcommand{\sectionmark}[1]{\markright{\thesection\enspace\spacedlowsmallcaps{#1}}}
%*******************************************************
% List of Figures and of the Tables
%*******************************************************
\clearpage

\begingroup 
    \let\clearpage\relax
    \let\cleardoublepage\relax
    \let\cleardoublepage\relax
%     %*******************************************************
%     % List of Figures
%     %*******************************************************    
%     %\phantomsection 
%     \refstepcounter{dummy}
%     %\addcontentsline{toc}{chapter}{\listfigurename}
%     \pdfbookmark[1]{\listfigurename}{lof}
%     \listoffigures
%
%     \vspace*{8ex}
%
%     %*******************************************************
%     % List of Tables
%     %*******************************************************
%     %\phantomsection 
%     \refstepcounter{dummy}
%     %\addcontentsline{toc}{chapter}{\listtablename}
%     \pdfbookmark[1]{\listtablename}{lot}
%     \listoftables
%         
%     \vspace*{8ex}
% %   \newpage
%     
%     %*******************************************************
%     % List of Listings
%     %*******************************************************      
% 	  %\phantomsection 
%     \refstepcounter{dummy}
%     %\addcontentsline{toc}{chapter}{\lstlistlistingname}
%     \pdfbookmark[1]{\lstlistlistingname}{lol}
%     \lstlistoflistings 
%
%     \vspace*{8ex}
       
    %*******************************************************
    % Acronyms
    %*******************************************************
    %\phantomsection 
    \refstepcounter{dummy}
    \pdfbookmark[1]{Initialisms and Abbreviations}{Initialisms and Abbreviations}
    \markboth{\spacedlowsmallcaps{Initialisms and Abbreviations}}{\spacedlowsmallcaps{Initialisms and Abbreviations}}
    \chapter*{Initialisms and Abbreviations}
    \begin{acronym}[AUROC]
        \acro{2AFC}[$2$AFC]{two-alternative forced-choice}
        \acro{ACh}[\ce{ACh}]{acetylcholine}
        \acro{AUROC}[AUROC]{area under \acl{ROC}\acroextra{ (\acs{ROC})} curve}
        \acro{BOLD}[BOLD]{blood oxygen-level dependent\acroextra{ contrast imaging}}
        \acro{CI}[CI]{confidence interval}
        \acro{cpd}[cpd]{cycles per degree}
        \acro{CRT}[CRT]{cathode ray tube}
        \acro{CSD}[CSD]{current source density}
        \acro{dva}[dva]{degrees of visual angle}
        %\acro{ECG}[ECG]{electrocardiogram}
        \acro{EEG}[EEG]{electroencephalography}
        %\acro{EPI}[EPI]{echo-planar imaging}
        \acro{FFT}[FFT]{fast Fourier transform}
        \acro{FIR}[FIR]{finite impulse response\acroextra{ filter}}
        %\acro{FOV}[FOV]{field of view}
        \acro{G}[G]{granular\acroextra{ compartment of \acs{V1}, equivalent to \acs{L4}}}
        \acro{IG}[IG]{infragranular\acroextra{ compartment of \acs{V1}, equivalent to \acs{L5/6}}}
        \acro{IIR}[IIR]{infinite impulse response\acroextra{ filter}}
        \acro{IT}[IT]{inferior temporal cortex\acroextra{ (Brodmann's Areas 20 and 21)}}
        \acro{KL}[KL]{Kullback-Leibler\acroextra{ divergence}}
        \acro{L}[L]{long\acroextra{ (``red'') cone}}
        \acro{L1}[L$1$]{layer $1$\acroextra{ of \acs{V1}}}
        \acro{L2/3}[L$2/3$]{layer $2/3$\acroextra{ of \acs{V1}}}
        \acro{L4}[L$4$]{layer $4$\acroextra{ of \acs{V1}, equivalent to \acs{G}}}
        \acro{L4Ca}[L$4$C\textalpha]{layer $4$C\textalpha\acroextra{ of \acs{V1}}}
        \acro{L4Cb}[L$4$C\textbeta]{layer $4$C\textbeta\acroextra{ of \acs{V1}}}
        \acro{L5}[L$5$]{layer $5$\acroextra{ of \acs{V1}}}
        \acro{L5A}[L$5$A]{layer $5$A\acroextra{ of \acs{V1}}}
        \acro{L5B}[L$5$B]{layer $5$B\acroextra{ of \acs{V1}}}
        \acro{L5/6}[L$5/6$]{layers $5$ and $6$\acroextra{ of \acs{V1}, equivalent to \acs{IG}}}
        \acro{L6}[L$6$]{layer $6$\acroextra{ of \acs{V1}}}
        %\acro{L6A}[L$6$A]{layer $6$A\acroextra{ of \acs{V1}}}
        %\acro{L6B}[L$6$B]{layer $6$B\acroextra{ of \acs{V1}}}
        \acro{LFP}[LFP]{local field potential}
        \acro{LGN}[LGN]{lateral geniculate nucleus}
        \acro{M}[M]{medium\acroextra{ (``green'') cone}}
        \acro{M1}[M1]{monkey 1}
        \acro{M2}[M2]{monkey 2}
        \acro{MEA}[MEA]{multi-electrode array}
        %\acro{MRI}[MRI]{magnetic resonance imaging}
        \acro{MSTd}[MSTd]{dorsal medial superior temporal area}
        \acro{MT}[MT]{middle temporal cortex\acroextra{, also known as \acs{V5}}}
        \acro{MUA}[MUA]{multi-unit activity}
        \acro{NaCl}[\ce{NaCl}]{sodium chloride}
        \acro{NH}[NH]{null hypothesis}
        %\acro{NMR}[NMR]{nuclear magnetic resonance}
        \acro{NSB}[NSB]{{N}emenman-{S}hafee-{B}ialek\acroextra{ entropy estimation method}}
        \acro{PFC}[PFC]{prefrontal cortex}
        \acro{PSTH}[PSTH]{peristimulus time histogram}
        \acro{PT}[PT]{{P}anzeri-{T}reves\acroextra{ bias correction method}}
        \acro{QE}[QE]{Quadratic Extrapolation\acroextra{ bias correction method}}
        \acro{R}[R]{rod\acroextra{ cell}}
        \acro{RF}[RF]{receptive field}
        \acro{RGC}[RGC]{retinal ganglion cell}
        \acro{ROC}[ROC]{receiver operating characteristic}
        \acro{S}[S]{short\acroextra{ (``blue'') cone}}
        %\acro{SEM}[SEM]{standard error on the mean}
        \acro{SG}[SG]{supragranular\acroextra{ compartment of \acs{V1}, equivalent to \acs{L1} and \acs{L2/3}}}
        \acro{SNR}[SNR]{signal-to-noise ratio}
        \acro{V1}[V1]{primary visual cortex\acroextra{ (Brodmann's Area 17)}}
        \acro{V2}[V2]{visual area 2\acroextra{ (Brodmann's Area 18)}}
        \acro{V3}[V3]{visual area 3}
        \acro{V4}[V4]{visual area 4}
        \acro{V5}[V5]{visual area 5\acroextra{, also known as \acl{MT} (\acs{MT})}}
        \acro{V6}[V6]{visual area 6\acroextra{, also known as dorsomedial area}}
    \end{acronym}                     
\endgroup

\cleardoublepage
