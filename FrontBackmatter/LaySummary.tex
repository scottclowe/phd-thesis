%*******************************************************
% Lay Summary
%*******************************************************

\pdfbookmark[1]{Lay Summary}{Lay Summary} % Bookmark name visible in a PDF viewer

\begingroup

\let\clearpage\relax
\let\cleardoublepage\relax
\let\cleardoublepage\relax

\chapter*{Lay Summary}
\addcontentsline{toc}{chapter}{\tocEntry{Lay Summary}}

The most complicated system known to man is that of his own brain.
It's often said that the human mind is the most powerful supercomputer on Earth, though this comparison can seem contrived as the two, brains and computers, clearly work in very different ways.
However, brains are, fundamentally, systems which process information about the world experienced through the senses (sight, hearing, touch, taste, smell, and others besides) and do computations so that we can extract meaning from this data --- distinguish the smell of a rose, tell the difference between a cat and a dog, recognise the face of a loved one.
% These things seem very easy to us, but
As we progress through the regions of the brain, moving from the parts directly connected to the sensory organs (eyes, ears, and so on), to the deeper recesses of the mind, representations within the brain become increasingly abstract.
Eventually the information about the world, now processed by other parts of the brain to pick out the really important bits, reach the regions of the brain involved in planning and decision making.


Since brains are information processing systems, we can study them using the tools of information theory to try to better understand how they function.
In this thesis, we study how the parts of the brain which process visual information work and allow us to see.
% We do this using the mathematical principles of information theory to determine which of the different components of the activity in the brain are related to .
When babies are born, their brains don't know how to handle the information from their eyes; they have to learn how to see.
Even as an adult, you can train your brain to form better representations of the things that you see.
If you repeatedly look at similar images and try to distinguish between them, you will get better with practice (though not forever --- at some point your performance will stop improving).
However, we don't know exactly what changes in the brain to enable you to do this.

We investigated this by tasking monkeys to distinguish between similar stimuli --- one image but presented with many different contrasts --- and recording the activity in their brains as they learnt to get better at this task.
We found that the first part of the brain (known as \acs{V1}) which processes vision was already very good at encoding the differences between the stimuli.
In fact, it was so good that it didn't need to get better than it was to begin with.
Another part of the brain (known as \acs{V4}), which analyses more abstract properties of the shapes of visual stimuli, initially didn't distinguish between the contrast of the stimuli.
But it got better with training, and the increase in information in this bit of the brain was the same as the increase in the performance of the monkey.
This suggests that the parts of the monkey's brain which make the decision about how to respond to the stimulus have to use the information in the latter part of the brain (\acs{V4}) and don't get to use the information which is in the first part (\acs{V1}).
One hypothesis is that this happens because \acs{V1} only has lots of information about these stimuli due to a quirk related to them being different contrasts.
% which isn't actually related with how the brain usually processes stimuli.
Stimuli in the real world vary in more important ways, and identifying the contrast of what you're seeing doesn't really help you to tell the difference between a bear and tree if you're out in the woods.
Only by training yourself on the task of contrast discrimination does your brain learn to focus on this, presumably less important, feature.


We then turned our attention to the oscillatory activity occurring in the part of the brain which first processes vision (\acs{V1}).
In the brain, the activity of neurons neighbouring each other within local regions fluctuate together in rhythmic harmony.
Importantly, the activity of the population can oscillate at more than one frequency at once.
To offer up an analogy, the neurons are like the players in an orchestra with violin, cello, and double bass sections.
The instruments play simultaneously and the high frequency oscillations of the violin (the high pitched notes) sit on top of the medium and slower oscillations of the cello and double bass (both lower pitched notes).
Except in the brain, every neuron can play multiple instruments at once.
Since there are lots of neurons, you can only hear one of the notes when the activity of many of the neurons are synchronised for the same note, otherwise its all just random noise.
The amplitude of these oscillations --- how loud the different notes are --- varies over time, and some of them are created by the neurons in response to the sensory input (\ie{} whatever the individual is looking at).

We studied how the amplitudes of the oscillations were triggered by different properties of natural stimuli by showing monkeys a clip from a Hollywood movie and recording the activity in their primary visual cortex (\acs{V1}).
The outside of your brain, which includes \acs{V1}, is made up of \num{6} layers stacked on top of each other, with each layer the thickness of a sheet of card.
We worked out which of the layers and which of the frequencies of oscillations contained information about the movie.
There are two different oscillations which encode information about the visual stimulus, and they correspond to different properties of the movie.
In particular, the low frequency oscillations relate to sudden, coarse, changes in the movie, which occur whenever there is a scene transition or jump cut.
This sort of change in stimulus is also like what happens when your eyes dart from one thing to another, so this signal may reflect how your brain copes with such sudden changes in visual stimulus.
The higher frequency oscillations relate to the finer details in the movie, like the edges of objects moving around.
Although the amplitude of the oscillations is, on average, the same in all the layers, only particular layers have oscillations which relate to the stimulus.
If we return to our orchestra analogy, this is like splitting our bassists into groups and observing that each group plays loudly and quietly some of the time.
All the groups play loudly as often as each other, but only one of the groups plays loudly when the movie they are accompanying moves from one scene to another.
Consequently, you can tell a when scene transition occurs just by listening to that group play together.
We don't know what causes the other groups to play loudly (or quietly), but we do know it isn't systematically related to the movie they're accompanying.

\endgroup
